\documentclass[presentation]{beamer}
\newcommand{\lectnum}{02}
\newcommand{\lectitle}{Oggetti e classi}
\usepackage{../oop-slides}

\begin{document}

\frame[label=coverpage]{\titlepage}

\ackpage{}

\newcommand{\ecl}{\eclipsepath{p02objects}}


\fr{Outline}{
  \bl{Goal della lezione}{\iz{
  \item Illustrare i concetti base del paradigma object-oriented
  \item Mostrare un primo semplice programma Java
  \item Fornire una panoramica di alcuni meccanismi Java
  }}
  \bl{Argomenti}{\iz{
  \item Oggetti e riferimenti
  \item Tipi primitivi
  \item Classi, metodi e campi
  \item Accenno a package e librerie
  \item Stampe a video
  \item Primo semplice programma Java
  }}
}

\section[Oggetti, tipi, valori]{Elementi base dei tipi di Java}

\fr{``Everything is an object''}{
  \bl{Riferimenti ad oggetti}{\iz{
  \item Nessun meccanismo per accedere ai dati per valore o puntatore!
  \item Le variabili conterranno dei riferimenti agli oggetti veri e propri, sono quindi dei nomi ``locali'' utilizzabili per denotare l'oggetto
  }}
  \sizedcode{\footnotesize}{code/references.java}
}

\fr{Variabili, oggetti, e valori primitivi}{
  \bl{Concetti base}{\iz{
  \item Variabile: un contenitore con nome (come in C), usabile per denotare un oggetto
  \item Valore primitivo: p.e. un numero, anche assegnabile ad una variabile
  }}
  \sizedcode{\footnotesize}{code/references2.java}
}


\fr{``(Almost) Everything is an object''}{
  \bl{Tipi primitivi: tipi speciali usati per valori atomici}{\iz{
  \item Assomigliano molto a quelli del C, ma hanno dimensioni fissate
  \item I \cil{boolean} possono valere \cil{true} o \cil{false}
  %\item Sono assegnabili a variabili, ma in questo caso lo sono ``per valore''
  %\item Ognuno ha una corrispondente classe chiamata \alert{wrapper}
  \item Altre classi di libreria (\cil{BigDecimal}, \cil{BigInteger}) gestiscono numeri di dimensione/precisione arbitraria
  }}
  \fg{width=0.7\textwidth}{img/primitive.png}
}

\fr{Una prima classificazione dei tipi}{
  \fg{width=1\textwidth}{img/types0.pdf}
}

\fr{Variabili e tipi}{
  \sizedcode{\footnotesize}{code/vars0.java}
}

\frs{5}{Costrutto \Cil{var}: ``local variable type inference'' (da Java 10)}{
  \bl{Costrutto \cil{var}}{\iz{
    \item usabile nelle variabili locali (a funzioni/metodi) per avere maggiore concisione
    \item il compilatore capisce (inferisce) il tipo della varibile dall'espressione assegnata
    \item non abusarne, e comunque noi non lo useremo molto all'inizio del corso
  }}
  \sizedcode{\footnotesize}{code/vars1.java}
}


\fr{Oggetti e memoria}{
  \bl{Gestione della memoria -- entriamo nella JVM solo temporanemante}{\iz{
    \item tutti gli oggetti sono allocati nella memoria \alert{heap}
    \item le variabili allocate nello stack, nei rispettivi record di attivazione
    \item le variabili di tipi primitivi contengono direttamente il valore
    \item le variabili che contengono oggetti in realtà hanno un riferimento verso lo heap
    \item nota: ancora non sappiamo cosa contiene un oggetto
  }}
  \bl{Situazione relativa al codice della slide precedente}{\iz{
    \item le variabili \cil{i}, \cil{d}, \cil{b}, \cil{other} contengono valori
    \item tutte le altre contengono un riferimento ad un oggetto, nello heap
    \item solo \cil{p} e \cil{q} ``(si) riferiscono (al)lo stesso oggetto''
    \item \cil{on} contiene un riferimento speciale, \cil{null}
  }}
}

\fr{Visibilità}{
  \bl{``Scope'' delle variabili}{\iz{
    \item \`E simile a quello del C
    \item variabili dentro un blocco non sono visibili fuori
    \item differenza rispetto al C: variabili non inizializzate non sono utilizzabili!
  }}
  \bl{Tempo di vita degli oggetti}{\iz{
    \item finito lo scope di una variabile, l'oggetto continua a esistere
    \item verrà deallocato automaticamente dal sistema se non più usato{\iz{
	\item se, direttamente o indirettamente, nessuna variabile lo può raggiungere
	\item un componente della JVM, il \alert{garbage collector}, è preposto a questo compito
	\item ne vedremo il funzionamento a breve..
    }}
  }}
}

\section[Costrutti OOP]{Principali costrutti dell'object-orientation}

\fr{Costruire classi}{
  \bl{Premesse}{\iz{
    \item la \alert{classe} è l'unità fondamentale di programmazione OO
    \item progettare e costruire classi correttamente sarà l'obbiettivo del corso
    \item incominciamo descrivendo la loro struttura generale
    \item solo nelle prossime lezioni daremo linee guida definitive
  }}
  \bl{Cos'è una classe}{\iz{
    \item è un template (stampino) per generare oggetti di una certa forma
    \item ne definisce tipo, struttura in memoria e comportamento
  }}
  \bl{Classe vs. oggetto}{\iz{
    \item classe: è una descrizione (parte di programma)
    \item oggetto: è una entità a tempo di esecuzione, è \alert{istanza} di una classe
  }}
  
}


\fr{Struttura di una classe}{
  \bl{Nome della classe}{
  è anche il nome del tipo
  }
  \bl{Proprietà della classe}{
    \alert{Campi} (o membri, o fields){\iz{
      \item descrivono la struttura/stato dell'oggetto in memoria
    }}
    \alert{Metodi} della classe{\iz{
      \item descrivono i messaggi accettati e il comportamento corrispondente
    }}
    (altri elementi che vedremo..)
   }
}

\fr{Classi: un po' di codice}{
  \titledcode{Costruzione classi}{code/classes1.java}
  \titledcode{Uso}{code/classes2.java}
}

\fr{Campi}{
  \bl{Elementi costitutivi dei campi}{\iz{
    \item i campi di una classe assomigliano ai membri di una struct del C
    \item ognuno è una sorta di variabile (nome + tipo)\\ (per i campi non è usabile il costrutto \cil{var})
    \item ve ne possono essere 0,1, molti
    \item lo stato di un oggetto è l'attuale valore associato ai campi
    \item potrebbero essere valori primitivi, o altri oggetti
  }}
  \bl{Valore di un campo}{\iz{
    \item impostabile al momento della sua dichiarazione
    \item se non inizializzato vale:
    {\iz{
      \item \cil{0} per i tipi numerici
      \item \cil{false} per i booleani
      \item \cil{null} per le classi
    }}
    \item accessibile da codice cliente con notazione \alert{obj.field}
  }}
  }
  
\fr{Campi: Un semplice esempio (toy)}{
  \titledcode{Classe}{code/fields1.java}
  \titledcode{Uso}{code/fields2.java}
}

\fr{Campi: l'esempio Point3D}{
  \titledcode{Classe}{code/fields_p3d_1.java}
  \titledcode{Uso}{code/fields_p3d_2.java}
}

\fr{Metodi}{
  \bl{Elementi costitutivi dei metodi}{\iz{
    \item i metodi di una classe assomigliano a funzioni (del C)
    \item ognuno ha una \alert{intestazione} (o signature) e un corpo{\iz{
     \item a sua volta l'intestazione ha il nome, tipo di ritorno, argomenti
    }}
    \item di metodi ve ne possono essere 0,1, molti
    \item definiscono il comportamento dell'oggetto
  }}
  \bl{Significato di un metodo}{\iz{
    \item codice cliente richiama un metodo con notazione \alert{obj.meth(args)}
    \item ciò corrisponde a inviargli un messaggio
    \item \cil{obj} è chiamato il \alert{receiver} del messaggio (o della invocazione)
    \item il comportamento conseguente è dato dall'esecuzione del corpo
    \item il corpo può leggere/scrivere il valore dei campi
  }}
  }
  
\fr{Metodi: Un esempio ``giocattolo'' (toy example)}{
  \titledcode{Classe}{code/methods1.java}
  \titledcode{Uso}{code/methods2.java}
}

\fr{La variabile speciale \Cil{this}}{
  \bl{\cil{this}}{\iz{
    \item dentro ad un metodo si può accedere agli argomenti o ai campi
    \item per rendere meno ambigua la sintassi, Java fornisce una variabile speciale denotata con \cil{this}, che contiene il riferimento all'oggetto che sta gestendo il messaggio
    \item per motivi di leggibilità, è opportuno usarla sempre
  }}
  \sizedcode{\scriptsize}{code/methods1this.java}
}


\fr{Metodi: altro esempio Point3D}{
  \sizedcode{\scriptsize}{code/methods_p3d_1.java}
  %\titledcode{Uso}{code/methods_p3d_2.java}
}

\section{Programmi Java}

\fr{Programmi Java}{
  \bl{Elementi costitutivi dei programmi Java}{\iz{
    \item librerie di classi del Java Development Kit
    \item librerie esterne (nostre o di altri)
    \item un insieme di classi che costituiscono l'applicazione
    \item almeno una di tali classi ha un metodo speciale \cil{main}
    \item un \cil{main} è il punto d'accesso di un programma
  }}
  \bl{Il main}{Il \cil{main} deve avere la seguente dichiarazione:{\iz{
  \item \cil{public static void main(String[] args)\{..\}}}}
  tre concetti che spiegheremo in dettaglio nel prosieguo:\iz{
  \item \cil{public} indica il fatto che debba essere visibile ``a tutti''
  \item \cil{static} indica che non è un metodo dell'oggetto, ma della classe
  \item \cil{String[]} indica il tipo ``array di stringhe''
  }}
}

\fr{Package e librerie Java: organizzazione}{
  \bl{Librerie di Java}{\iz{
    \item Documentazione auto-generata, consultabile offline, o online: \\ \myurl{https://docs.oracle.com/en/java/javase/11/docs/api/} \\ (google search ``\texttt{javadoc 11}'')
    \item contano più di 4000 classi, raggruppate in 200+ \alert{package} (e 50+ \alert{moduli})
  }}
  \bl{Package}{\iz{
    \item un package è un contenitore di classi con uno scopo comune, di alto livello
    \item tipicamente un package contiene qualche decina di classi
    \item i package sono organizzati ad albero, con notazione \cil{nome1.nome2.nome3}
    \item Package principali: \cil{java.lang}, \cil{java.util}, \cil{java.io}
  }}
}

\frs{5}{Package, moduli e librerie Java: moduli}{
  \bl{Moduli (Java 9+)}{\iz{
    \item un modulo definisce un frammento di codice ``autonomo'':{\iz{
    \item testabile, distribuibile, con chiara interfaccia e dipendenze da altri
    }}
    \item p.e., \cil{java.base}, \cil{java.desktop}, \cil{java.sql}, \cil{java.xml}
    \item librerie esterne compatibili con Java 9+ sono distribuite in uno o più moduli
    \item un modulo è costituito internamente da uno o più package
    \item p.e., \cil{java.base} contiene i package principali che useremo    
  }}
  \bl{Impatto sulla programmazione ``base''}{\iz{
    \item il concetto di modulo \alert{non impatta} i sorgenti, ma solo il ``project management'': per il momento non ce ne occuperemo perché il JDK fornisce ``di default'' accesso a tutti i moduli che ci servono (\cil{java.*})
    \item il concetto di package \alert{invece impatta} i sorgenti: il nome completo di una classe dipende dal package in cui si trova 
  }}
}


\frs{5}{Package, moduli e librerie Java: uso}{
  \bl{Importare una classe di ``libreria''}{\iz{
    \item per usare le classi di una libreria prima le si importa
    \item lo si fa con la clausola \cil{import}, da usare all'inizio del sorgente
    {\iz{
      \item importo la singola classe: \cil{import java.util.Date;}
      \item importo l'intero package: \cil{import java.util.*;}
      \item importazione di default: \cil{import java.lang.*;}
    }}
    \item senza importazioni, ogni classe andrebbe sempre qualificata indicandone anche il package completo: \cil{java.util.Date obj = new java.util.Date();}
    \item importare evita quindi solo di dover indicare ogni volta il package
  }}
  \bl{Classi (e funzionalità) ``deprecate''}{\iz{
    \item dichiarate come ``scadute'', ossia preferibilmente ``da non usare più'' (legacy)
    \item noi le utilizzeremo per scopi didattici
    \item p.e. \cil{java.util.Date}
  }}
}

\fr{Stampe su schermo}{
  \bl{La procedura di stampa \cil{System.out.println}}{\iz{
    \item \cil{System} è una classe nel package \cil{java.lang}
    \item \cil{out} è un suo campo statico, rappresenta lo standard output
    \item \cil{println} è un metodo che accetta una stringa e la stampa
    \item l'operatore \cil{+} concatena stringhe a valori
    %\item \cil{format} è un analogo metodo usabile in stile ``\cil{printf}''
  }}
  \code{code/print.java}
}

\fr{``Hello world'' Java Program}{
  \titledcode{Hello.java}{exec/Hello.java}
  \bl{Compilazione ed esecuzione}{\iz{
    \item con un editor di testo si scrive la classe in un file \cil{Hello.java}
    \item si compila la classe col comando: \cil{javac Hello.java}
    \item se non ci sono errori, viene generato il \alert{bytecode} \cil{Hello.class}
    \item si esegua il programma con: \cil{java Hello}
    \item la JVM cerca la classe \cil{Hello}, e ne esegue il \cil{main}
    
  }}
}

\fr{Librerie, oggetti, e stampe}{
  \sizedcode{\scriptsize}{exec/PrintingObjects.java}
}

\fr{Il caso delle classi Wrapper dei tipi primitivi}{
   \bl{Le classi Wrapper}{\iz{
      \item ce ne è una per tipo primitivo, nel package \cil{java.lang}{\iz{
         \item \cil{Integer}, \cil{Double}, \cil{Boolean},...
      }}
      \item hanno qualche metodo utile per fare conversioni
      \item poco usate per il momento
   }}
   \srcode{\ssmall}{3}{100}{\ecl/Wrappers.java}
}

\fr{Una nuova classificazione dei tipi}{
  \fg{width=1\textwidth}{img/types.pdf}
}

\fr{Costruire e provare classi}{
  \sizedcode{\scriptsize}{exec/Point3D.java}
}

\fr{Classi, e classi clienti}{
  \sizedcode{\tiny}{code/Point3D_a.java}
  \sizedcode{\tiny}{exec/UsePoint3D.java}
  \vspace{-10pt}\bx{\iz{
  \item si compilano separatamente, o con: \cil{javac *.java}
  \item si esegue a partire dalla classe col \cil{main}: \cil{java UsePoint3D}
  }}
}

\fr{Preview del prossimo laboratorio}{
  \bl{Obbiettivi}{\iz{
    \item familiarizzare con la compilazione da linea di comando in Java
    \item fare qualche esercizio con la costruzione e uso di classi
  }}
}





\end{document}