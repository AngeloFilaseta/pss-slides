\documentclass[presentation]{beamer}
\newcommand{\lectnum}{12}
\newcommand{\lectitle}{Errori di programmazione ed Exceptions}
\usepackage{../oop-slides}

\begin{document}

\frame[label=coverpage]{\titlepage}

\ackpage{}


\newcommand{\ecl}{\eclipsepath{p12exceptions}}
\newcommand{\ecls}{\eclipsepath{p12exceptions/safedevices}}


\fr{Outline}{
  \bl{Goal della lezione}{\iz{
  \item Illustrare i vari meccanismi di gestione delle eccezioni in Java
  \item Dare linee guida per la progrettazione di sistemi che usano eccezioni
  }}
  \bl{Argomenti}{\iz{
  \item Errori a run-time e necessità di una loro gestione
  \item Tipi di eccezioni/errori in Java
  \item Istruzione \cil{throw}
  \item Costrutto \cil{try-catch-finally}
  \item Dichiarazioni \cil{throws}
  }}
}

\section{Errori}

\fr{Errori nei programmi}{
  \bl{Errori a tempo di compilazione (compile-time)}{\iz{
    \item sono quelli più grossolani, sono intercettati dal compilatore
    \item quindi rientrano nella fase dell'implementazione, sono innocui
    \item un linguaggio con strong typing consente di identificarne molti a compile-time
  }}
  \bl{Errori a tempo di esecuzione (run-time) ($\Leftarrow$ oggetto della lezione)}{\iz{
    \item sono condizioni anomale dovute alla dinamica del sistema{\iz{
      \item parametri anomali a funzioni, errori nell'uso delle risorse di sistema,..
    }}
    \item in genere è possibile (i) identificare/descrivere dove potrebbero accadere, (ii) intercettarli e (iii) gestirli prevedendo procedure di compensazione (rimedio al problema che le ha causate)
    \item alcuni linguaggi (come Java, non il C) forniscono costrutti per agevolarne la gestione
  }}
}

\fr{Errori per causa interna: lanciati dalla JVM}{
  \titledcode{Errore numerico}{code/Div.java}
  \titledcode{Overflow memoria}{code/Rec.java}
  \titledcode{Riferimento null}{code/Null.java}
}

\fr{Violazioni di contratto d'uso di un oggetto: librerie Java}{
  \titledcode{Operazione non supportata}{code/Op.java}
  \titledcode{Elemento non disponibile}{code/NoElement.java}
  \titledcode{Formato illegale}{code/Format.java}
}

\fr{Violazioni di contratto d'uso di un oggetto: nostro codice}{
  \titledcode{Argomento errato}{code/Arg.java}
  \titledcode{Elemento non disponibile}{code/It.java}
}

\fr{L'importanza della ``error-aware programming''}{
  \bl{Contratti}{\iz{
    \item Molti oggetti richiedono determinate condizioni di lavoro (sequenze di chiamata, argomenti passati, aspettative d'uso di risorse computazionali)
    \item Al di fuori queste condizioni è necessario interrompere il lavoro e effettuare azioni correttive
  }}
  \bl{Il progettista della classe deve:}{\en{
    \item identificare le condizioni di lavoro definite ``normali''
    \item intercettare quando si esce da tali condizioni
    \item eventualmente segnalare l'avvenuto errore
  }}
  \bl{Il cliente (a sua volta progettista di un altro oggetto) deve:}{\en{
    \item essere informato di come l'oggetto va usato
    \item intercettare gli errori e porvi rimedio con un \it{handler}
  }}
}

\fr{Eccezioni in Java}{
  \bl{Riassunto Java Exceptions}{\iz{
    \item Gli errori a run-time in Java sono rappresentati da oggetti della classe \cil{java.lang.Throwable}. Vengono ``lanciati'':{\iz{
      \item da esplicite istruzioni del tipo: \cil{throw <exception-object>;}
      \item o, direttamente dalla JVM per cause legate al ``sistema operativo''
    }}
    \item Tali oggetti portano informazioni utili a capire la causa dell'errore
    \item Si può dichiarare se un metodo potrà lanciare una eccezione: \cil{<meth-signature> throws <excep-class>\{..\}}
    \item Si può intercettare una eccezione e porvi rimedio: \cil{try\{ <instructions> \} catch(<excep-class> <var>)\{...\}}
  }}
  \bx{{\large Tutti meccanismi che impareremo a progettare e implementare in questa lezione!}}
}

\fr{Tipologie di errori in Java}{
  \bl{Errori: \cil{java.lang.Error} e sottoclassi}{\iz{
    \item Dovute a un problema ``serio'' (e non risolvibile) interno alla JVM
    \item Di norma una applicazione non si deve preoccupare di intercettarli (non ci sarebbe molto di più da fare che interrompere l'esecuzione)
  }}
  \bl{Eccezioni unchecked: \cil{java.lang.RuntimeException} e sottoclassi}{\iz{
    \item Causate da un bug nella programmazione
    \item Di norma una applicazione non si deve preoccupare di intercettarli (dovrebbero essere risolti tutti in fase di debugging del sistema)
  }}
  \bl{Eccezioni checked: i \cil{java.lang.Throwable} tranne le precedenti}{\iz{
    \item Causate da un problema prevedibile ma non rimediabile a priori
    \item Le applicazione devono dichiarli esplicitamente, e vanno intercettati e gestiti esplicitamente
  }}
}

\fr{Tipologie di errori in Java: UML}{
  \fg{height=0.85\textheight}{img/uml-throw.pdf}
}

\fr{Usuale gestione}{
  \bl{Errori}{\iz{
    \item Nessuna gestione necessaria (``se capitano, capitano...'')
  }}
  \bl{Eccezioni unchecked}{\iz{
    \item Si potrebbero dichiarare con un commento al codice
    \item Di norma si riusano le classi \cil{java.lang.RuntimeException} del JDK, ossia non se ne definiscono di nuove tipologie
    \item Si lanciano con l'istruzione \cil{throw}
  }}
  \bl{Eccezioni checked}{\iz{
    \item Vanno dichiarate nel metodo con la clausola \cil{throws}
    \item La documentazione deve spiegare in quali casi vengono lanciate
    \item Vanno intercettate con l'istruzione \cil{try-catch}
    \item Di norma si costruiscono sotto-classi ad-hoc di \cil{Exception}
  }}
}

\fr{Errori ed eccezioni unchecked: cosa accade}{
  \bl{Quando accadono, ossia quando vengono lanciate..}{\iz{
    \item Causano l'interruzione dell'applicazione
    \item Comportano la scrittura su console di errore (\cil{System.err}) di un messaggio che include lo \alert{StackTrace} -- \cil{Thread.dumpStack();}{\iz{
      \item nota, solitamente \cil{System.err} coincide con \cil{System.out}
    }}
    \item Dal quale possiamo desumere la sequenza di chiamate e il punto del codice in cui si ha avuto il problema
  }}
  \bl{Errori/eccezioni unchecked comuni e già viste}{\iz{
    \item \cil{StackOverFlowError}: stack esaurito (ricorsione infinita?)
    \item \cil{NullPointerException}, \cil{ArrayStoreException}, \cil{ClassCastException}, \cil{ArrayIndexOutOfBoundsException}, \cil{NumericException}, \cil{OperationNotSupportedException}
    \item Altri andranno verificati sulla documentazione quando incontrati
  }}
}

\fr{Esempio di stampa}{
  \sizedrangedcode{\ssmall}{3}{30}{\ecl/classes/UncheckedStackTrace.java}
}

\fr{L'istruzione \Cil{throw}}{
  \sizedrangedcode{\ssmall}{3}{30}{\ecl/classes/UncheckedThrow.java}
}

\fr{L'istruzione \Cil{throw}: una variante equivalente}{
  \sizedrangedcode{\ssmall}{3}{30}{\ecl/classes/UncheckedThrow2.java}
}

\fr{L'istruzione \Cil{throw}}{
  \bl{Sintassi generale}{
    \cil{throw <expression-that-evaluates-to-a-throwable>;}
  }
  \bl{Casi tipici}{
    \cil{throw new <exception-class>(<message-string>);}\\
    \cil{throw new <exception-class>(<ad-hoc-args>);}\\
    \cil{throw new <exception-class>();}\\
  }
  \bl{Effetto}{\iz{
    \item Si interrompe immediatamente l'esecuzione del metodo in cui ci si trova (se non dentro una \cil{try-catch}, come vedremo dopo..)
    \item L'oggetto eccezione creato viene ``riportato'' al chiamante
    \item Ricorsivamente, si giunge al \cil{main}, con la stampa su \cil{System.err} (exception chaining)
  }}
}

\section[Es. \texttt{RangeIterator}]{Impl.corretta di \texttt{RangeIterator}}

\fr{Riconsideriamo l'implementazione di \Cil{RangeIterator}}{
  \bl{Elementi da considerare}{\iz{
    \item Controllare l'interfaccia \cil{java.util.Iterator}
    \item Verificare la documentazione presente nel sorgente (ed in particolare, come si specificano le eccezioni lanciate)
    \item Il comando: \cil{javadoc Iterator.java}
    \item La documentazione HTML prodotta
    \item Realizzazione e prova di \cil{RangeIterator}
  }}
}

\fr{Documentazione di \Cil{Iterator}: header}{
  \sizedcode{\ssmall}{code/util/Iterator1.java}
}

\fr{Documentazione di \Cil{Iterator}: \Cil{next()} e \Cil{hasNext()}}{
  \sizedcode{\ssmall}{code/util/Iterator2.java}
}

\fr{Documentazione di \Cil{Iterator}: \cil{remove()}}{
  \sizedcode{\ssmall}{code/util/Iterator3.java}
}

\fr{Documentazione generata: pt1}{
  \fg{height=0.9\textheight}{img/javadoc1.png}
}

\fr{Documentazione generata: pt2}{
  \fg{height=0.9\textheight}{img/javadoc2.png}
}

\fr{Realizzazione di \Cil{RangeIterator}}{
    \vspace{-5pt}\sizedrangedcode{\ssmall}{3}{100}{\ecl/classes/RangeIterator.java}
}

\section{Catturare eccezioni}

\fr{Il costrutto \Cil{try-catch}}{
  \bl{Sintassi (da estendere successivamente)}{
    \cil{try \{ <body-maybe-throwing-an-exception>\}} \\
    ~~~~\cil{catch (<throwable-class> <var>) \{ <handler-body>\}}\\
  }
  \bl{Esempio}{
    \cil{try \{ RangeIterator r = new RangeIterator(a,b);\}} \\
    ~~~~\cil{catch (RuntimeException e) \{ System.out.println(e);\}}\\
  }
  \bl{Significato}{\iz{
    \item Se il body nella \cil{try} lancia una eccezione del tipo specificato nella \cil{catch}
    \item Allora si esegue il corrispondente handler, e non si ha la terminazione della applicazione
    \item Se non c'è eccezione si salta l'handler e si prosegue
  }}
}

\fr{Uso della \Cil{RangeIterator} senza \Cil{try-catch}}{
  \sizedrangedcode{\ssmall}{3}{30}{\ecl/classes/UseRange.java}
}

\fr{Uso della \Cil{RangeIterator} con \Cil{try-catch}}{
  \sizedrangedcode{\scriptsize}{3}{30}{\ecl/classes/UseRange2.java}
}

\fr{Il costrutto \Cil{try-catch-finally}}{
  \bl{Sintassi generale}{
    \cil{try \{ <body-maybe-throwing-an-exception>\}} \\
    ~~~~\cil{catch (<throwable-class> <var>) \{ <handler-body>\}}\\
    ~~~~\cil{catch (<throwable-class> <var>) \{ <handler-body>\}}\\
    ~~~~...\\
    ~~~~\cil{catch (<throwable-class> <var>) \{ <handler-body>\}}\\
    ~~~~\cil{finally \{ <completion-body>\} // clausola finale opzionale}\\
  }
  \bl{Significato}{\iz{
    \item Se il body nella \cil{try} lancia una eccezione
    \item La prima \cil{catch} pertinente esegue l'handler (non ci possono essere sovrapposizioni!)
    \item Poi si eseguirà anche il \cil{completion-body}
    \item Il body nella \cil{finally} sarà comunque eseguito!
  }}
}

\fr{\Cil{catch} multipli e \Cil{finally}}{
  \sizedcode{\scriptsize}{code/classes/UseRange3.java}
}

\fr{Spiegazione}{
  \bl{Come funziona la \cil{finally}?}{\iz{
    \item garantisce che il codice nel suo handler sarà sicuramente eseguito
    \item ..sia se ho avuto eccezione
    \item ..sia se non ho avuto eccezione
    \item ..sia se uno degli handler delle catch ha generato eccezione
  }}
  \bl{A cosa serve?}{\iz{
    \item in genere contiene del codice di cleanup che deve comunque essere eseguito
    \item rilascio risorse, chiusura file, stampa messaggi, etc..
  }}
  \bl{Vedremo la prossima settimana il costrutto chiamato \cil{try}-with-resources}{\iz{
    \item consente di non esprimere il \cil{finally}
  }}
}

\section[Creare eccezioni]{Creazione e rilancio eccezioni}

\fr{Creazione di una nuova classe di eccezioni}{
  \bl{Nuove eccezioni}{\iz{
    \item Un sistema potrebbe richiedere nuovi tipi di eccezioni, che rappresentano eventi specifici collegati al dominio applicativo{\iz{
      \item Persona già presente (in un archivio cittadini)
      \item Lampadina esaurita (in una applicazione domotica)
    }}
    \item Semplicemente si fa una estensione di \cil{Exception} o \cil{RuntimeException}{\iz{
      \item a seconda che la si voglia checked o unchecked
      \item per il momento stiamo considerando solo le unchecked
    }}
    \item Non vi sono particolari metodi da ridefinire di solito
    \item Solo ricordarsi di chiamare correttamente il costruttore del padre
    \item Se si vuole incorporare una descrizione articolata della causa dell'eccezione, la si può inserire nei campi dell'oggetto tramite il costruttore o metodi setter..
  }}
}

\fr{Esempio: \Cil{MyException}}{
  \sizedrangedcode{\scriptsize}{3}{30}{\ecl/classes/MyException.java}
}

\fr{Esempio: \Cil{UseMyException}}{
  \sizedrangedcode{\ssmall}{3}{30}{\ecl/classes/UseMyException.java}
}

\section[Definire eccezioni]{Dichiarazione eccezioni checked}

\fr{Checked vs Unchecked}{
  \bl{Unchecked: \cil{RuntimeException} o sottoclassi}{\iz{
    \item Quelle viste finora, dovute ad un bug di programmazione
    \item Quindi sono da catturare opzionalmente, perché rimediabili
    \item[$\Rightarrow$] ..le linee guida più moderne le sconsigliano
  }}
  \bl{Checked: \cil{Exception} o sottoclassi ma non di \cil{RuntimeException}}{\iz{
    \item Rappresentano errori non riconducibili ad una scorretta programmazione, ma ad eventi abbastanza comuni anche nel sistema una volta installato e funzionante
    {\iz{
      \item Funzionamento non normale, ma non tale da interrompere l'applicazione (p.e., l'utente fornisce un input errato inavvertitamente)
      \item Un problema con l'interazione col S.O. (p.e., file inesistente)
    }}
    \item I metodi che le lanciano lo \alert{devono} dichiararle esplicitamente (\cil{throws})
    \item Chi chiama tali metodi \alert{deve} obbligatoriamente gestirle{\iz{
      \item o catturandole con un \cil{try-catch}
      \item o rilanciandole al chiamante con la \cil{throws} 
    }}
  }}
}

\fr{Una eccezione checked: \Cil{IOException} e input da tastiera}{
  \sizedrangedcode{\scriptsize}{3}{30}{\ecl/classes/IOFromKeyboard.java}
}

\fr{Qualche variante: campi statici}{
  \sizedrangedcode{\scriptsize}{3}{30}{\ecl/classes/IOFromKeyboard2.java}
}

\fr{Qualche variante: input iterato e rilancio}{
  \sizedrangedcode{\ssmall}{3}{30}{\ecl/classes/IOFromKeyboard3.java}
}

\fr{Input da tastiera: da Java 6 (non funziona in Eclipse!)}{
  \sizedrangedcode{\scriptsize}{3}{30}{\ecl/classes/IOFromKeyboard4.java}
}

\section[Esempio]{Applicazione domotica con eccezioni}

\fr{Requirements}{
  \bx{\iz{
    \item Una fila di $n$ \cil{Device} con tempo di vita limitato
    \item Il sistema dovrà supportare in futuro diverse politiche di fine-vita dei device
    \item Il fine-vita viene rilevato al tentativo di accensione, ed è segnalato da una eccezione checked
    \item Esistono comandi per accendere e spegnere tutti i device
    \item Il sistema dovrà essere a prova di qualunque eccezione
  }}
}

\fr{UML: Modellazione \Cil{Device}}{
  \fg{height=0.8\textheight}{img/uml-device.pdf}
}

\fr{Interfaccia \Cil{Device}}{
  \sizedrangedcode{\ssmall}{10}{100}{\ecls/Device.java}
}

\fr{Eccezione \Cil{DeviceIsOverException}}{
  \sizedrangedcode{\scriptsize}{7}{100}{\ecls/DeviceIsOverException.java}
}

\fr{\Cil{AbstractDevice}, pt1}{
  \sizedrangedcode{\ssmall}{3}{26}{\ecls/AbstractDevice.java}
}

\fr{\Cil{AbstractDevice}, pt2}{
  \sizedrangedcode{\ssmall}{28}{100}{\ecls/AbstractDevice.java}
}

\fr{\Cil{CountdownDevice}}{
  \sizedrangedcode{\scriptsize}{3}{100}{\ecls/CountdownDevice.java}
}

\fr{\Cil{DeviceRow}: Campi e costruttore}{
  \sizedrangedcode{\ssmall}{5}{32}{\ecls/DeviceRow.java}
}

\fr{\Cil{DeviceRow}: Selettori e \Cil{allOff()}}{
  \sizedrangedcode{\ssmall}{34}{57}{\ecls/DeviceRow.java}
}

\fr{\Cil{DeviceRow}: \Cil{allOn()} e \Cil{toString()}}{
  \sizedrangedcode{\ssmall}{58}{100}{\ecls/DeviceRow.java}
}

\fr{\Cil{UseDevice}}{
  \sizedrangedcode{\ssmall}{3}{100}{\ecls/UseDevice.java}
}

\fr{Una applicazione completa}{
  \bl{Elementi aggiuntivi}{\iz{
    \item Vi dovrà essere una interazione con l'utente
    \item Potrà da console fornire i comandi: \cil{+N}, \cil{-N}, \cil{+all}, \cil{-all}, \cil{exit}
    \item ..e vedere direttamente l'effetto che hanno sul \cil{DeviceRow}
    \item (è un prodromo di applicazione con GUI..)
  }}
  \bl{Un problema architetturale}{\iz{
    \item come allestire una applicazione con interazione con l'utente?
  }}
}

\fr{Il pattern architetturale MVC}{
  \bl{MVC -- divide l'applicazione in 3 parti}{\iz{
    \item \cil{Model}: modello OO del dominio applicativo del sistema
    \item \cil{View}: gestisce le interazioni con l'utente (in futuro una GUI)
    \item \cil{Controller}: gestisce il coordinamento fra Model e View
  }}
  \bl{Applicazione (domotica)}{\iz{
    \item \cil{Model}{\iz{
	\item Un wrapper per un \cil{DeviceRow}
    }}
    \item \cil{View}{\iz{
      \item Implementata da un \cil{ConsoleView} che lavora con la Console
    }}
    \item \cil{Controller}{\iz{
	\item Utilizza un \cil{CommandExecutor} che ``processa'' i comandi da tastiera
    }}
    \item[$\Rightarrow$] \cil{View} e \cil{Model} nascoste da interfacce, per supportare un buon disaccoppiamento
  }}
  
}

\fr{UML: Design generale}{
  \fg{height=0.8\textheight}{img/dia.jpg}
}

\fr{\Cil{Controller}, pt1}{
  \sizedrangedcode{\ssmall}{3}{27}{\ecls/Controller.java}
}

\fr{\Cil{Controller}, pt2}{
  \sizedrangedcode{\ssmall}{29}{100}{\ecls/Controller.java}
}

\fr{Eccezioni}{
  \sizedrangedcode{\scriptsize}{3}{100}{\ecls/ExitCommandException.java}
  \sizedrangedcode{\scriptsize}{3}{100}{\ecls/CommandNotRecognisedException.java}
}

\fr{Interfaccia e implementazione \Cil{Model}}{
    \sizedrangedcode{\ssmall}{3}{100}{\ecls/Model.java}
    \sizedrangedcode{\ssmall}{7}{100}{\ecls/DeviceRowModel.java}
}


\fr{Interfaccia \Cil{View}}{
  \sizedrangedcode{\ssmall}{3}{100}{\ecls/View.java}
}


\fr{\Cil{ConsoleView}, pt1}{
  \sizedrangedcode{\ssmall}{5}{28}{\ecls/ConsoleView.java}
}

\fr{\Cil{ConsoleView}, pt2}{
  \sizedrangedcode{\ssmall}{29}{100}{\ecls/ConsoleView.java}
}

\fr{Interfaccia \Cil{CommandExecutor}}{
  \sizedrangedcode{\scriptsize}{3}{100}{\ecls/CommandExecutor.java}
}


\fr{\Cil{SimpleCommandExecutor}, pt1}{
  \sizedrangedcode{\ssmall}{4}{25}{\ecls/SimpleCommandExecutor.java}
}

\fr{\Cil{SimpleCommandExecutor}, pt2}{
  \sizedrangedcode{\ssmall}{26}{50}{\ecls/SimpleCommandExecutor.java}
}

\fr{\Cil{SimpleCommandExecutor}, pt3}{
  \sizedrangedcode{\ssmall}{51}{67}{\ecls/SimpleCommandExecutor.java}
}

\fr{\Cil{SimpleCommandExecutor}, pt4}{
  \sizedrangedcode{\ssmall}{68}{100}{\ecls/SimpleCommandExecutor.java}
  }
  
\fr{Note su questa progettazione}{
  \bl{Complessivamente}{\iz{
    \item è lungi dall'essere ottimale
    \item è un primo passo verso l'idea di ``buon progetto''
  }}
  \bl{Aspetti positivi}{\iz{
    \item Suddivisione base secondo logica MVC
    \item M e V ``nascosti'' da interfacce, favorendo disaccoppiamento
  }}
  \bl{Aspetti da migliorare -- ve ne sono sempre!!}{\iz{
    \item \cil{Controller} contiene elementi relativi all'interazione con l'utente{\iz{
      \item sarebbero da astrarre in chiamate di metodo da fare sulla \cil{View}
    }}
    \item \cil{CommandExecutor} contiene due logiche, e non andrebbe bene:{\iz{
      \item riconoscimento della stringa in input (da gestire nella \cil{View})
      \item conseguente esecuzione del comando
    }} 
  }}
  
}


\end{document}

