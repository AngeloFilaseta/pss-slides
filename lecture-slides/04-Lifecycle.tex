\documentclass[presentation]{beamer}
\usepackage{oop-slides}

\title[OOP04: Oggetti e Classi pt 2]{04 \\ Oggetti e Classi pt.2\\ inizializzazione, accessi, distruzione}

\begin{document}

\newcommand{\ecl}{\eclipsepath{p04objects}}

\frame[label=coverpage]{\titlepage}

\fr{Outline}{
  \bl{Goal della lezione}{\iz{
  \item Completare i meccanismi OO di Java a livello di classe
  \item Porre le basi per poter fare della corretta progettazione
  \item Discutere aspetti collegati alla vita di un oggetto (costruzione, accesso, distruzione)
  }}
  \bl{Argomenti}{\iz{
  \item Codice statico
  \item Costruttori
  \item Overloading di metodi
  \item Approfondimento sui package
  \item Controllo d'accesso
  \item Finalizzazione e garbage collection
  }}
}

\section{Codice statico}

\frs{5}{Codice statico}{
  \bl{Meccanismo}{\iz{
  \item Alcuni campi e metodi di una classe possono essere dichiarati \cil{static}
  \item Esempi visti:{\iz{
    \item il \cil{main} dal quale parte un programma Java
    \item il metodo \cil{java.util.Arrays.toString()} (ossia, il metodo \cil{toString} nella classe \cil{Arrays} dentro al package \cil{java.util})
    \item il campo \cil{java.lang.System.out}
    \item il metodo \cil{java.lang.System.getProperty()}
  }}
  \item Tali campi e metodi sono considerate proprietà della classe, non dell'oggetto
  \item Sintassi: \cil{classe.metodo(argomenti)}, e \cil{classe.campo}
  {\iz{\item omettendo la classe si assume quella da cui parte la chiamata}}
  }}
  \bl{Sulla notazione}{\iz{
    \item Iniziano con minuscolo: nomi di package, metodi e campi
    \item Iniziano con maiuscolo: nomi di classe
    \item[$\Rightarrow$] Questo consente di capire facilmente il significato delle istruzioni
  }}
}

\fr{Uso di codice statico o non statico in una classe \Cil{C}?}{
  \bl{Codice non statico (detto anche codice istanza)}{\iz{
  \item È codice puro object-oriented
  \item Definisce le operazioni e lo stato di ogni oggetto creato da \cil{C}
  }}
  \bl{Codice statico}{\iz{
  \item Non è codice puro object-oriented, ma codice in stile ``imperativo/strutturato''
  \item Definisce funzioni e variabili del componente software definito da \cil{C}{\iz{
    \item possono essere visti come metodi e campi dell'unico oggetto ``classe \cil{C}'' 
  }}
  }}
  \bl{Usarli assieme?}{\iz{
    \item Le helper class abbiano solo metodi/campi statici
    \item Le classi OO non abbiano metodi/campi statici, a meno di qualche funzionalità ``generale'' della classe (si veda nel proseguio)
  }}
}

\fr{Esempio Point3D}{
    \srcode{\ssmall}{3}{100}{\ecl/Point3D.java}
}

\fr{Uso Point3D}{
    \srcode{\ssmall}{3}{100}{\ecl/UsePoint3D.java}
}

\fr{Point3D: commenti}{
    \bl{Si notino le due diverse chiamate}{\iz{
      \item \cil{build} è sempre chiamata su un oggetto, ossia su: \cil{new Point3D()}
      \item \cil{zero} e \cil{max} sono chiamate sulla classe \cil{Point3D}
    }}
    \bl{Razionale}{\iz{
      \item \cil{max} è una funzionalità sui \cil{Point3D}, quindi è naturale includerla nella classe \cil{Point3D}, ma non è inquadrabile come funzionalità di un singolo oggetto (ognuno darebbe la stessa risposta)
      \item \cil{zero} è trattato come ``costante'', quindi non è proprietà di un oggetto
    }}
    \bl{Un errore comune}{\iz{
      \item se si omette l'indicazione del receiver, si rischia di chiamare una proprietà non statica da un metodo statico, e questo comporta un errore segnalato dal compilatore
      \item[$\Rightarrow$] a questo si ovvia inserendo sempre l'indicazione del receiver
    }}
}

\fr{Variante Point3DBis.. inizializzazione con ritorno}{
    \srcode{\ssmall}{3}{100}{\ecl/Point3DBis.java}
}

\fr{Uso Point3DBis}{
    \srcode{\ssmall}{3}{100}{\ecl/UsePoint3DBis.java}
}

\fr{Point3DBis: commenti}{
    \bl{Si noti la struttura del metodo \cil{build}}{\iz{
      \item inizializza il valore dei campi
      \item restituisce \cil{this}
    }}
    \bl{Razionale}{\iz{
      \item è possibile concatenare creazione di oggetto e sua inizializzazione
      \item schema chiamato ``fluente''
    }}
}

\fr{Sull'uso delle proprietà \Cil{static}}{
    \bl{Consigli generali}{\iz{
      \item nella programmazione OO pura, andrebbero interamente evitati 
      \item frequente invece l'uso di costanti
      \item a volte sono comodi, per ora cercare di tenere separate nelle classi le parti \cil{static} dalle altre (poi vedremo ``pattern'' più sofisticati)
      \item le librerie di Java ne fanno largo uso
    }}
    \bl{Prassi frequente in Java}{\iz{
	\item se \cil{XYZ} è una classe usata per generare oggetti, la classe \cil{XYZs} conterrà solo proprietà statiche relative
	\item es: \cil{Object}/\cil{Objects}, \cil{Array}/\cil{Arrays}, \cil{Collection}/\cil{Collections}
    }}
    \bl{In altri linguaggi.. come Scala}{\iz{
	\item parte statica e non-statica \alert{vanno necessariamente} posizionate in porzioni diverse del codice
    }}
}


\fr{Ristrutturazione \Cil{Point3D}}{
    \srcode{\scriptsize}{3}{100}{\ecl/points/Point3D.java}
}

\fr{Modulo \Cil{Points}}{
    \srcode{\scriptsize}{3}{100}{\ecl/points/Points.java}
}

\fr{Uso \Cil{Point3D} e \Cil{Points}}{
    \srcode{\scriptsize}{3}{100}{\ecl/points/UsePoint3D.java}
}


%\fr{Inizializzazione statica}{
%    \bl{Problema}{\iz{
%      \item A volte la inizializzazione dei campi statici non è comodamente eseguibile con un semplice assegnamento
%      \item ..come nel caso del campo \cil{Point3D.zero}
%      \item Si usa l'inizializzatore statico. Sintassi: \cil{static \{ ... \}}
%      \item Viene eseguito al momento del caricamento della classe
%    }}
%    \sizedcode{\scriptsize}{04/fragments/Point3Dstatic.java}
%}

\section{Costruttori}

\fr{La costruzione degli oggetti}{
    \bl{L'operatore \cil{new}}{\iz{
      \item Allo stato attuale delle nostre conoscenze, crea un oggetto inizializzando tutti i suoi campi al loro valore di default (p.e. $0$ per i numeri), e ne restituisce il riferimento
      \item Altri metodi devono gestire la inizializzazione vera e propria
      \item Problema: garantire una corretta inizializzazione
    }}
    \bl{Costruttori di una classe}{\iz{
      \item Assomigliano per struttura ai metodi
      \item Hanno lo stesso nome della classe in cui si trovano
      \item Nessun tipo di ritorno, possono avere dei parametri formali{\iz{
      \item[$\Rightarrow$] alla \cil{new} si possono quindi passare dei valori
      }}
      \item Il costruttore di default (a zero argomenti) è implicitamente definito solo se non se ne aggiungono altri -- ecco perché era consentito scrivere: \cil{new Point3D()}
    }}
}

\fr{Esempio: Point3D}{
  \code{04/fragments/Point3Dcons.java}
}

\fr{Gli argomenti del costr. spesso corrispondono ai campi}{
  \code{04/fragments/Point3Dvar.java}
}

\fr{Esempio: Persona (3 costruttori)}{
  %\sizedcode{\scriptsize}{04/code/Persona.java}
  \srcode{\ssmall}{3}{100}{\ecl/Persona.java}
}

\fr{Esempio: Uso di Persona (3 costruttori)}{
  \srcode{\scriptsize}{3}{100}{\ecl/UsePersona.java}
  \bl{Sequenza d'azioni effettuate con una \cil{new}}{\iz{
      \item si crea l'oggetto con tutti i campi inizializzati al default
      \item si esegue il codice del costruttore (\cil{this} punta all'oggetto creato)
      \item la \cil{new} restituisce il riferimento \cil{this}
  }}
}

\fr{Istruzione \Cil{this(..)}}{
  \bl{Usabile per chiamare un altro costruttore}{\iz{
    \item tale istruzione può solo essere la prima di un costruttore
    \item questo meccanismo consente di riusare il codice di altri costruttori
  }}
  \srcode{\ssmall}{3}{100}{\ecl/Persona2.java}
}

\fr{Overloading dei costruttori}{
  \bl{Overloading: un nome, più significati}{\iz{
    \item L'overloading è un meccanismo importante per il programmatore
    \item La difficoltà che comporta è dovuta alla tecnica di disambiguazione
    \item Due esempi visti finora: overloading operatori matematici, overloading costruttori
  }}
  \bl{Overloading dei costruttori}{
    Data una \cil{new}, quale costruttore richiamerà? (JLS 15.12)
    \iz{
      \item si scartano i costruttori con numero di argomenti che non corrisponde
      \item si scartano i costruttori il cui tipo d'argomenti non è compatibile
      \item se ve ne è uno allora è quello che verrà chiamato..
      \item altrimenti il compilatore segnala un errore
  }}
}

\fr{Overloading dei metodi}{
  \bl{Overloading dei metodi}{\iz{
    \item Si può fare overloading dei metodi con stessa tecnica di risoluzione
    \item Sia su metodi statici, che su metodi standard (metodi istanza)
  }}
  \srcode{\tiny}{3}{100}{\ecl/ExampleOverloading.java}
}

\section{Controllo d'accesso}

\fr{I package}{
 \bl{Problema}{\iz{
   \item Un framework mainstream come quello di Java può disporre  di decine di migliaia di classi di libreria e di applicazione
   \item \`E necessario un meccanismo per consentire di strutturarle in gruppi (a più livelli gerarchici)
   \item Per poter meglio gestirli in memoria secondaria, e per meglio accedervi
 }}
 \bl{Package in Java}{\iz{
  \item Ogni classe appartiene ad un \alert{package}
  \item Un package ha nome gerarchico \cil{n1.n2....nj} (p.e. \cil{java.lang})
  \item Le classi di un package devono trovarsi in una medesima directory
  \item Questa è la subdirectory \cil{n1/n2/../nj} a partire da una delle directory presenti nella variabile d'ambiente \cil{CLASSPATH}
 }}
}

\fr{I package pt. 2}{
 \bl{Dichiarazione del package}{\iz{
   \item Ogni unità di compilazione (file \cil{.java}) deve specificare il suo package
   \item Lo fa con la direttiva \cil{package pname;}
   \item Se non lo fa, allora trattasi del package di ``default''
   \item Se lo fa, tale file dovrà stare nella opportuna directory
 }}
 \bl{Importazione classi da altri package}{\iz{
  \item Una classe va usata (p.e. nella \cil{new} o come tipo) specificando anche il package (p.e. \cil{new java.util.Date()})
  \item Per evitare tale specifica, si inserisce una direttiva di importazione \iz{
    \item \cil{import java.util.*;} importa tutte le classi di \cil{java.util}
    \item \cil{import java.util.Date;} importa la sola \cil{java.util.Date}
  }
  
 }}
}

\fr{Unità di compilazione e livello d'accesso ``package''}{
 \bl{Unità di compilazione}{\iz{
   \item \`E un file compilabile in modo atomico da \cil{javac}
   \item Si deve chiamare con estensione \cil{.java}
   \item Può contenere varie classi indicate una dopo l'altra 
 }}
 \bl{Livello d'accesso package}{\iz{
   \item Le classi in una unità di compilazione, i loro metodi, campi, e costruttori hanno di default il \alert{livello d'accesso package}
   \item[$\Rightarrow$] sono visibili e richiamabili solo dentro al package stesso
   \item[$\Rightarrow$] sono invisibili da fuori
 }}
}

\fr{Livelli d'accesso \Cil{public} e \Cil{private}}{
 \bl{Livello d'accesso \cil{public} -- visibile da tutte le classi}{\iz{
   \item Lo si indica anteponendo alla classe/metodo/campo/costruttore la keyword \cil{public}
   \item La corrispondente classe/metodo/campo/costruttore sarà visibile ed utilizzabile da qualunque classe, senza limitazioni
 }}
 \bl{Livello d'accesso \cil{private} -- visibile solo nella classe corrente}{\iz{
   \item Lo si indica anteponendo al metodo/campo/costruttore la keyword \cil{private}
   \item Il corrispondente metodo/campo/costruttore sarà visibile ed utilizzabili solo dentro alla classe in cui è definito
 }}
}

\fr{Qualche conseguenza}{
 \bl{A livello di classe}{\iz{
   \item In una unità di compilazione solo una classe può essere \cil{public}, e questa deve avere lo stesso nome del file \cil{.java}
 }}
 \bl{A livello di metodi/campi/costruttori}{\iz{
   \item la scelta \cil{public}/\cil{private} può consentire di gestire a piacimento il concetto di \alert{information hiding}, come approfondiremo la prossima settimana
 }}
}

\fr{La keyword \Cil{final}}{
 \bl{\cil{final} = non modificabile}{\iz{
   \item Ha vari utilizzi: in metodi, campi, argomenti di funzione e variabili
   \item Tralasciamo per ora il caso dei metodi
   \item Negli altri casi denota variabili/campi assegnati e non più modificabili
 }}
 \bl{Il caso più usato: costanti di una classe}{\iz{
    \item Es.: \cil{public static final int CONST=10;}
    \item Perché usarle? {\iz{
    \item Anche se \cil{public}, si ha la garanzia che nessuno le modifichi
    \item Sono gestibili in modo più performante
      }}
 }}
 \bl{Il caso dei ``magic number''}{\iz{
    \item Ogni numero usato in una classe per qualche motivo (3,21,..)..
    \item ..sarebbe opportuno venisse mascherato da una costante, per motivi di leggibilità e di più semplice modificabilità
 }}
}

\fr{Esempio Magic Numbers}{
  \srcode{\footnotesize}{3}{100}{\ecl/MagicExample.java}
}


\fr{\Cil{GuessMyNumberApp} revisited}{
  \srcode{\tiny}{3}{100}{\ecl/GuessMyNumberApp.java}
}

\section{Distruzione oggetti}

\fr{Allocazione degli oggetti in memoria}{
  \bl{Operatore \cil{new}}{\iz{
    \item Incorpora l'equivalente della funzione \cil{malloc} del C
    \item Il compilatore calcola la dimensione necessaria da allocare
    \item La \cil{new} chiama il gestore della memoria, che alloca lo spazio necessario
    \item Si inizializza l'area e si restituisce il suo riferimento
  }}
  \bl{Cosa c'è in un oggetto (e, quanto  è grande?)}{\iz{
    \item Non tutti i dettagli sono noti, dipende dalla JVM
    \item Di sicuro contiene i seguenti elementi:\iz{
      \item Spazio per ogni campo (non statico) della classe
      \item Un riferimento ad una struttura dati relativa alla classe dell'oggetto
      \item Un riferimento alla \alert{tabella dei metodi virtuali} (una struttura dati necessaria a trovare dinamicamente i metodi da richiamare)
  }}}
}

\fr{Distruzione degli oggetti}{
  \bl{Il tempo di vita degli oggetti}{\iz{
    \item Durante l'esecuzione di un programma, è verosimile che molto oggetti vengano creati
    \item Ogni creazione comporta l'uso di una parte di memoria centrale
    \item Non è noto quanto durerà l'esecuzione del programma
    \item[$\Rightarrow$] Qualcuno dovrà preoccuparsi di deallocare la memoria
  }}
  \bl{Il garbage collector (GC)}{\iz{
    \item E' un componente della JVM richiamato dalla JVM con una frequenza che dipende dallo stato della memoria
    \item Ogni volta, cerca oggetti in memoria heap che nessuna parte attiva del programma (thread) sta più usando (neanche indirettamente)
    \item Trovatili, li dealloca (come la \cil{free} del C) \alert{senza che il programmatore debba occuparsene}
  }}
}

\fr{Esempio funzionamento GC}{
  \srcode{\ssmall}{3}{100}{\ecl/GC.java}
}


\section{Una applicazione}

\frs{10}{Applicazione: \Cil{Mandelbrot}}{
  \bl{Problema}{
  Data una semplice classe \cil{Picture} che gestisce gli aspetti grafici, realizzare una applicazione che disegna il frattale Mandelbrot.
  }
  \bl{Elementi progettuali}{\iz{
    \item Classe fornita \cil{Picture} -- codice non comprensibile ora{\iz{
      \item ha un costruttore che accetta larghezza e altezza in pixel della finestra
      \item metodo \cil{void drawPixel(int x,int y,int color)}
    }}
    \item Classe \cil{Complex} modella numeri complessi e operazioni base
    \item Classe \cil{Mandelbrot} si occupa di calcolare il valore di ogni punto del rettangolo{\iz{
      \item metodo \cil{void advancePosition()} passa al prossimo punto
      \item metodo \cil{boolean isCompleted()} dice se ci sono altri punti da calcolare
      \item metodo \cil{int computeIteratrions()} dice quante iterazioni vengono calcolate per il punto corrente
    }}
    \item Classe \cil{MandelbrotApp} ha il solo \cil{main}
  }}
  \bl{Elementi implementativi}{\iz{
    \item Implementazione ancora preliminare e da migliorare
  }}
}

\fr{Classe \Cil{Complex}}{
  \srcode{\ssmall}{3}{100}{\ecl/app/Complex.java}
}

\fr{Classe \Cil{Mandelbrot}}{
  \srcode{\tiny}{3}{100}{\ecl/app/Mandelbrot.java}
}

\fr{Classe \Cil{MandelbrotApp}}{
  \srcode{\tiny}{4}{100}{\ecl/app/MandelbrotApp.java}
}


\fr{Preview del prossimo laboratorio}{
  \bl{Obbiettivi}{\iz{
    \item Esercizi su piccoli algoritmi su array e tipi primitivi
    \item Uso costruttori
    \item Aspetti avanzati della compilazione in Java
  }}
}




\end{document}