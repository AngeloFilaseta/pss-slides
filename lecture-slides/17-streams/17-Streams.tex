\documentclass[presentation]{beamer}
\newcommand{\lectnum}{17}
\newcommand{\lectitle}{Java 8: Streams}
\usepackage{../oop-slides}

\begin{document}

\frame[label=coverpage]{\titlepage}

\ackpage{}


\newcommand{\ecl}{\eclipsepath{p17streams}}


\fr{Outline}{
  \bl{Goal della lezione}{\iz{
  \item Mostrare la gestione funzionale degli Stream
  \item Discutere altri aspetti relativi alle novità di Java 8
  }}
}

\section{Stream}

\frs{5}{Il concetto di Stream}{
  \bl{Idee}{\iz{
    \item Uno Stream rappresenta un flusso sequenziale (anche infinito) di dati omogenei, usabile una volta sola, e dal quale si vuole ottenere una informazione complessiva e/o aggregata
    \item Assomiglia al concetto di Iteratore, ma lo Stream è più dichiarativo, perché non indica passo-passo come l'informazione viene processata, e quindi è concettualmente più astratto
    \item Ove possibile, uno Stream manipola i suoi elementi in modo ``lazy'' (ritardato): i dati vengono processati mano a mano che servono, non sono memorizzati tutti assieme come nelle Collection
    \item E' possibile creare ``catene'' di trasformazioni di Stream (implementate con decorazioni successive) in modo funzionale, per ottenere flessibilmente computazioni non banali dei loro elementi, con codice più compatto e leggibile
    \item Questa modalità di lavoro rende le computazioni (automaticamente) parallelizzabili, ossia computabili da un set arbitrario di Thread
  }}
}

\frs{10}{Computazioni con gli Stream}{
  \bl{Struttura a pipeline}{\iz{
    \item Una sorgente o sink:{\iz{
      \item Una Collection/array, un dispositivo di I/O, una funzione generatrice
    }}
    \item Una sequenza di trasformazioni:{\iz{
      \item mappe e filtri, ma non solo..
    }}
    \item Un terminatore, che aggrega i dati dello Stream:{\iz{
      \item una riduzione ad un valore, una Collection/array, un Iteratore
    }}
  }}
  \bl{Esempio: con persone con nome, città e reddito}{\iz{
    \item Data una \cil{List<Persona>} con proprietà reddito e città, ottenere la somma dei redditi di tutte le persone di Cesena
    \item Come lo realizziamo tramite una pipeline?{\iz{
      \item Sorgente: la lista
      \item Trasformazione 1: filtro sulle persone di Cesena
      \item Trasformazione 2: si mappa ogni persona sul suo reddito
      \item Terminazione: sommo
    }}
    \item Aspetto cruciale: le fasi intermedie (dopo le trasformazioni), non generano collezioni temporanee
  }}
}

\fr{Classe \Cil{Person} -- \Cil{equals}, \Cil{hashCode} e \Cil{toString} omessi}{
    \sizedrangedcodet{\tiny}{5}{34}{\ecl/Person.java}
}


\fr{Realizzazione dell'esempio in Java 8}{
    \sizedrangedcodet{\tiny}{5}{100}{\ecl/UseStreamsOnPerson.java}
}

\fr{La libreria degli Stream}{
  \bl{Struttura}{\iz{
    \item Package \cil{java.util.stream}: interfacce e classi per gli stream
    \item Interfaccia \cil{Stream<X>}: stream e metodi statici di ``factory''
    \item Interfaccia \cil{BaseStream<X,B>}: sopra-interfaccia di \cil{Stream} con i metodi base
    \item Interfaccia \cil{DoubleStream}: stream di \cil{double}, con metodi base e specifici
    \item Interfacce \cil{IntStream}, \cil{LongStream}: simili
    \item Interfaccia \cil{Collector<T,A,R>}: rappresenta una operazione di riduzione
    \item Classe \cil{Collectors}: fornisce una serie di collettori
    \item ..altre classi di Java creano degli stream
  }}
}

\fr{Le collection generano Stream!}{
    \sizedrangedcode{\scriptsize}{1}{100}{snippets/Collection.java}
}

\fr{L'interfaccia \Cil{java.util.BaseStream}}{
    \sizedrangedcode{\ssmall}{1}{100}{snippets/BaseStream.java}
}

\frs{5}{Riassunto delle funzionalità di una pipeline per \Cil{Stream<X>}}{
    \bl{Creazione}{\iz{
      \item \cil{empty}, \cil{of}, \cil{iterate}, \cil{generate}, \cil{concat}
    }}
    \bl{Trasformazione}{\iz{
      \item \cil{filter}, \cil{map}, \cil{flatMap}, \cil{distinct}, \cil{sorted}, \cil{peek}, \cil{limit}, \cil{skip}, \cil{mapToXYZ},..
    }}
    \bl{Terminazione}{\iz{
      \item \cil{forEach}, \cil{forEachOrdered}, \cil{toArray}, \cil{reduce}, \cil{collect}, \cil{min}, \cil{max}, \cil{count}, \cil{anyMatch}, \cil{allMatch}, \cil{noneMatch}, \cil{findFirst}, \cil{findAny},..
    }}
    \bl{Una nota sulle classi \cil{DoubleStream} e simili}{\iz{
      \item sono più specializzate e performanti, non avendo il boxing
      \item non hanno tutte le funzionalità di cui sopra, se vi servono vi dovete riportare ad un \cil{Stream<X>} con un trasformatore \cil{mapToObj()} o \cil{boxed()}
      \item ne hanno qualcuna in più specifica, ad esempio \cil{sum}
    }}
}


\fr{\Cil{java.util.Stream}: costruzione stream, 1/3}{
    \sizedrangedcode{\tiny}{1}{26}{snippets/Stream.java}
}

\fr{\Cil{java.util.Stream}: trasformazione stream, 2/3}{
    \sizedrangedcode{\tiny}{26}{45}{snippets/Stream.java}
}

\fr{\Cil{java.util.Stream}: terminazione stream, 3/3}{
    \sizedrangedcode{\tiny}{46}{100}{snippets/Stream.java}
}

\fr{Trasformazioni di Stream: esempi}{
    \sizedrangedcode{\ssmall}{5}{100}{\ecl/UseTransformations.java}
}

\fr{Trasformazioni di Stream: esempi pt.2}{
    \sizedrangedcode{\ssmall}{5}{100}{\ecl/UseTransformations2.java}
}

\fr{Creazione di Stream: esempi}{
    \sizedrangedcode{\ssmall}{6}{100}{\ecl/UseFactories.java}
}

\fr{Creazione di Stream: esempi pt.2}{
    \sizedrangedcode{\ssmall}{5}{100}{\ecl/UseFactories2.java}
}

\fr{Creazione di Stream: file di testo e stringhe}{
    \sizedrangedcode{\tiny}{6}{100}{\ecl/UseOtherFactories.java}
}

\fr{Terminazioni ad-hoc di Stream: esempi}{
    \sizedrangedcode{\tiny}{5}{100}{\ecl/UseTerminations.java}
}

\fr{Terminazione generalizzata con \Cil{Stream.collect}}{
    \sizedrangedcodet{\tiny}{6}{100}{\ecl/UseGeneralizedCollectors.java}
}

\fr{Collettori di libreria: la classe \Cil{Collectors}}{
    \sizedrangedcodet{\tiny}{1}{100}{snippets/Collectors.java}
}

\fr{\Cil{UseCollectors}: collettori di base}{
    \sizedrangedcode{\ssmall}{4}{100}{\ecl/UseCollectors.java}
}

\fr{\Cil{UseCollectors}: collettori di base pt 2}{
    \sizedrangedcode{\tiny}{4}{100}{\ecl/UseCollectors2.java}
}

\fr{Esempi avanzati su \Cil{Person}}{
    \sizedrangedcodet{\tiny}{7}{100}{\ecl/UseStreamsOnPerson2.java}
}

  \fr{Algoritmi funzionali -- cosa realizzano?}{
    \sizedrangedcodet{\tiny}{7}{100}{\ecl/UseStreamsForAlgorithms.java}
}

\section{Implementazione Stream e Concorrenza}

\fr{Stream e parallelismo}{
  \bl{Uno dei vantaggi degli stream}{\iz{
    \item E' possibile gestire in modo completamente automatico il parallelismo
    \item Alcuni stream (ossia non tutti) ammettono implementazioni multi-threaded{\iz{
      \item \cil{range}, collezioni.. ma non \cil{iterate}
    }}
    \item Gli effettivi vantaggi (o addirittura penalizzazioni!) dipendono da molti fattori{\iz{
      \item Implementazione corrente dell'API, tipo di stream, tipi di computazioni
    }}
  }}
  \bl{Come si abilita il parallelismo (ove possibile)?}{\iz{
    \item Basta richiamare il metodo \cil{parallel()} su uno stream..
    \item ..o il metodo \cil{parallelStream()} da una collection
  }}
}

\fr{Test concurrency}{
    \sizedrangedcodet{\ssmall}{8}{100}{\ecl/TestConcurrency.java}
}

\fr{Considerazioni sul guadagno di performance -- dati molto variabili..}{
  \bl{Nel caso specifico (PC quad core)}{\iz{
    \item Con $1'000'000$ elementi nella collection, rapporto circa $0.65$ (più lento)
    \item Con $10'000'000$ elementi nella collection, circa $1.1$
    \item Con $100'000'000$ elementi nella collection, circa $3.22$
  }}
  \bl{Indicazioni generali}{\iz{
    \item Usare parallelismo solo con istanze di dimensione significativa
    \item E' bene se le lambda usate nelle trasformazioni sono semplici
    \item Verificate sempre prima di usare il parallelismo se conviene
    
  }}
}

\fr{Ma come sono implementati internamente questi stream?}{
  \bl{Elementi}{\iz{
    \item Qualche informazione è deducibile velocemente dal codice sorgente dell'API
    \item Gli stream sorgente incapsulano uno \cil{Spliterator} -- un \cil{Iterator} con funzionalità aggiuntive per supportare il parallelismo
    \item Gli stream trasformatori sono decoratori degli stream a monte -- vedere \cil{java.util.stream.AbstractPipeline}
    \item Lo stream a valle innesca la chiamata a catena degli elementi negli stream precedenti, uno alla volta (possibilmente in modo lazy).
  }}
  \bl{Una considerazione}{\iz{
    \item Sarebbe interessante potersi costruire le proprie classi per gli stream, ma la maggior parte delle funzionalità di basso livello utili/necessarie non sono pubbliche
  }}
}

\fr{Interfaccia Spliterator}{
  \sizedrangedcode{\tiny}{1}{100}{snippets/Spliterator.java}
}

\fr{Esempio: \Cil{Streams.RangeSpliterator}}{
  \sizedrangedcode{\tiny}{1}{100}{snippets/RangeSpliterator.java}
}


\end{document}
