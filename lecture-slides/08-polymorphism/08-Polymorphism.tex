\documentclass[presentation]{beamer}
\newcommand{\lectnum}{08}
\newcommand{\lectitle}{Polimorfismo (inclusivo, con le classi)}
\usepackage{../oop-slides}

\begin{document}

\newcommand{\ecl}{\eclipsepath{p08polymorphism}}
%\srcode{\tiny}{3}{32}{\ecl/Point3D.java}

\frame[label=coverpage]{\titlepage}

\ackpage{}


\fr{Outline}{
  \bl{Goal della lezione}{\iz{
  \item Illustrare la connessione fra polimorfismo inclusivo e ereditarietà
  \item Mostrare le interconnessioni con interfacce e classi astratte
  \item Mostrare le varie ripercussioni nel linguaggio
  }}
  \bl{Argomenti}{\iz{
  \item Polimorfismo inclusivo con le classi
  \item Layout oggetti in memoria
  \item Autoboxing dei tipi primitivi
  \item Tipi a run-time (cast, \cil{instanceof})
  \item Classi astratte
  }}
}

\section{Polimorfismo inclusivo con le classi}

\fr{Ereditarietà e polimorfismo}{
  \bl{Ricordando il principio di sostituibilità}{
    Se \texttt{B} è un sottotipo di \texttt{A} allora ogni oggetto di \texttt{B} può essere utilizzato (e ``deve poter'' essere utilizzabile) dove ci si attende un oggetto di \texttt{A}
  }
  \bl{Con l'ereditarietà}{\iz{
    \item Con la definizione: \cil{class B extends A\{..\}}
    \item Gli oggetti della classe \cil{B} rispondo a tutti i messaggi previsti dalla classe \cil{A}, ed eventualmente a qualcuno in più
    \item Quindi un oggetto della classe \cil{B} potrebbe essere passato dove se ne aspetta uno della classe \cil{A}, senza comportare problemi (di ``typing'')
  }}
  \bl{Conseguenza:}{
    Visto che è possibile, corretto, ed utile, allora in Java si considera \cil{B} come un sottotipo di \cil{A} a tutti gli effetti!
  }
}

\fr{Polimorfismo con le classi}{
\fg{width=0.7\textwidth}{img/uml-poli.pdf}
}

\fr{Polimorfismo con le classi}{
  \bl{In una classe \cil{D} che usi una classe \cil{C}...}{\iz{
    \item ci saranno punti in \cil{D} in cui ci si attende oggetti della classe \cil{C}
    \item (come argomenti a metodi, o da depositare nei campi)
    \item si potranno effettivamente passare oggetti della classe \cil{C}, ma anche delle classi \cil{C1}, \cil{C2},..,\cil{C5}, o di ogni altra classe successivamente creata che estende, direttamente o indirettamente \cil{C}
  }}
  \bl{Le sottoclassi di \cil{C}}{A tutti gli effetti, gli oggetti delle sottoclassi di \cil{C} sono compatibili con gli oggetti della classe \cil{C}\iz{
    \item hanno medesimo contratto (in generale, qualche operazione in più)
    \item hanno tutti i campi definiti in \cil{C} (in generale, qualcuno in più)
    \item hanno auspicabilmente un comportamento compatibile
  }}
}

%\section{Layout oggetti in memoria}

%\fr{La compatibilità delle sottoclassi}{
%  \sizedcode{\ssmall}{\ecl/various/ObjLayout.java}
%}

%\fr{Layout oggetti in memoria}{
%\fg{width=1\textwidth}{img/layout.pdf}
%}

\fr{Layout oggetti in memoria}{
  \bl{Alcuni aspetti del layout degli oggetti in memoria...}{
    Diamo alcune informazioni generali e astratte. Ogni JVM potrebbe realizzare gli oggetti in modo un po' diverso. Questi elementi sono tuttavia comuni.
  }
  \bl{Struttura di un oggetto in memoria}{\iz{
    \item Inizia con una intestazione ereditata da \cil{Object} (16 byte circa), che include {\iz{
      \item Indicazione di quale sia la classe dell'oggetto (runtime type information)
      \item Tabella dei puntatori ai metodi, per supportare il late-binding
      \item I campi (privati) della classe \cil{Object}
    }}
    \item Via via tutti i campi della classe, a partire da quelli delle superclassi
  }}
  \bl{Conseguenze: se la classe \cil{C} è sottoclasse di \cil{A}...}{
    Allora un oggetto di \cil{C} è simile ad un oggetto di \cil{A}, ha solo informazioni aggiuntive in fondo, e questo semplifica la sostituibilità!
  }
}

\fr{Esempio applicazione polimorfismo fra classi -- UML}{
  \fg{height=0.8\textheight}{img/uml-person.pdf}
}

\fr{\Cil{Person}}{
  \sizedrangedcode{\scriptsize}{3}{30}{\ecl/person/Person.java}
}

\fr{\Cil{Student}}{
  \sizedrangedcode{\scriptsize}{3}{30}{\ecl/person/Student.java}
}

\fr{\Cil{Teacher}}{
  \sizedrangedcode{\scriptsize}{3}{30}{\ecl/person/Teacher.java}
}

\fr{\Cil{UsePerson}}{
  \sizedrangedcode{\scriptsize}{3}{30}{\ecl/person/UsePerson.java}
  
}

%\fr{\Cil{LimitedLamp}}{
%  \sizedcode{\ssmall}{\ecl/classes2/LimitedLamp.java}
%}
%
%\fr{\Cil{LimitCounter}}{
%  \sizedcode{\scriptsize}{\ecl/classes2/LimitCounter.java}
%}
%
%\fr{\Cil{UnlimitedCounter}}{
%  \sizedcode{\scriptsize}{\ecl/classes2/UnlimitedCounter.java}
%}
%
%\fr{\Cil{UseLimitedLamp}}{
%  \sizedcode{\scriptsize}{\ecl/classes2/UseLimitedLamp.java}
%}


\fr{La differenza col caso del polimorfismo con le interfacce}{
  \bl{Polimorfismo con le interfacce}{\iz{
    \item La classe \cil{D} usa una interfaccia \cil{I}, non un'altra classe \cil{C}
    \item Si può assumere vi sia un certo contratto, ma non che vi sia uno specifico comportamento (quello di \cil{C}) che sia stato eventualmente specializzato
  }}
  \bl{Le classi non consentono ``ereditarietà multipla'' (in C++ si)}{\iz{
    \item *NON* è possibile in Java dichiarare: \cil{class C extends D1,D2 ...}{\iz{
      \item si creerebbero problemi se \cil{D1} e \cil{D2} avessero proprietà comuni
      \item diventerebbe complicato gestire la struttura in memoria dell'oggetto
    }}
    \item Con le interfacce non ci sono questi problemi, risultato:{\iz{
      \item è molto più semplice prendere una classe esistente e renderla compatibile con una interfaccia \cil{I}, piuttosto che con una classe \cil{C}
      }}
  }}
}

\fr{Riassunto polimorfismo inclusivo}{
  \bl{Polimorfismo}{\iz{
    \item Fornisce sopratipi che raccolgono classi uniformi tra loro
    \item Usabili da funzionalità ad alta riusabilità
    \item Utile per costruire collezioni omogenee di oggetti
  }}
  \bl{Polimorfismo con le interfacce}{\iz{
    \item Solo relativo ad un contratto
    \item Facilità nel far aderire al contratto classi esistenti
    \item Spesso vi è la tendenza a creare un alto numero di interfacce
  }}
  \bl{Polimorfismo con le classi}{\iz{
    \item Relativo a contratto e comportamento
    \item In genere ci si aderisce per costruzione dall'inizio
    \item Vincolato dall'ereditarietà singola
  }}
}

\fr{Come simulare ereditarietà multipla?}{
  \fg{height=0.8\textheight}{img/uml-diamond.pdf}
}

\frs{25}{Come simulare ereditarietà multipla?}{
  \bl{Definizioni}{\iz{
    \item \cil{interface Counter\{...\}}
    \item \cil{interface MultiCounter extends Counter\{...\}}
    \item \cil{interface BiCounter extends Counter\{...\}}
    \item \cil{interface BiAndMultiCounter extends MultiCounter,BiCounter\{...\}}
    \item \cil{class CounterImpl implements Counter\{...\}}
    \item \cil{class MultiCounterImpl extends CounterImpl implements MultiCounter\{...\}}
    \item \cil{class BiCounterImpl extends CounterImpl implements BiCounter\{...\}}
    \item \cil{class BiAndMultiCounterImpl extends BiCounterImpl implements BiAndMultiCounter\{...\}}
  }}
  \bl{Implementazione di \cil{BiAndMultiCounterImpl}}{\iz{
    \item si estende da \cil{BiCounterImpl}, si delega via composizione ad un oggetto di \cil{MultiCounterImpl}
  }}
  \bl{Complessivamente}{\iz{
    \item si ha completa e corretta sostituibilità tramite le interfacce
    \item si ha ottimo riuso delle implementazioni
    \item[$\Rightarrow$] si esplori la possibilità di usare solo delegazione, non ereditarietà
  }}
}



\section{Tipi a run-time}

\fr{Everything is an \Cil{Object}}{
  \bl{Perché avere una radice comune per tutte le classi?}{\iz{
    \item Consente di fattorizzare lì comportamento comune
    \item Consente la costruzione di funzionalità che lavorano su qualunque oggetto
  }}
  \bl{Esempi di applicazione:}{\iz{
    \item Container polimorfici, ad esempio via array di tipo \cil{Object[]}{\iz{
      \item permette di costruire un elenco di oggetti di natura anche diversa
      \item \cil{new Object[]\{new SimpleLamp(),new Integer(10)\}}
    }}
    \item Definizione di metodi con numero variabile di argomenti{\iz{
      \item argomenti codificati come \cil{Object[]}
    }}
  }}
}

\fr{Uso di \Cil{Object[]}}{
  \srcode{\footnotesize}{5}{30}{\ecl/last/AObject.java}
}

\fr{Tipo statico e tipo a run-time}{
  \bl{Una dualità introdotta dal subtyping (polimorfismo inclusivo)}{\iz{
    \item Tipo statico: il tipo di dato di una espressione desumibile dal compilatore
    \item Tipo run-time: il tipo di dato del valore(/oggetto) effettivamente presente (potrebbe essere un sottotipo di quello statico){\iz{\item in questo caso le chiamate di metodo sono fatte per late-binding}}
  }}
  \bl{Esempio nel codice di \cil{printAll()}, dentro al \cil{for}}{\iz{
    \item Tipo statico di \cil{o} è \cil{Object}
    \item Tipo run-time di \cil{o} varia di volta in volta: \cil{Object},\cil{String},\cil{Integer},..
  }}
  \bl{Ispezione tipo a run-time}{\iz{
    \item In alcuni casi è necessario ispezionare il tipo a run-time
    \item Lo si fa con l'operatore \cil{instanceof}
  }}
}


\fr{\Cil{instanceof} e conversione di tipo}{
  \srcode{\scriptsize}{3}{30}{\ecl/last/AObject2.java}
}

\fr{\Cil{instanceof} e conversione di tipo}{
  \bl{Ispezione ed uso della sottoclasse effettiva}{
    Data una variabile (o espressione) del tipo statico \cil{C} può essere necessario capire se sia della sottoclasse \cil{D}, in tal caso, su tale oggetto si vuole richiamare un metodo specifico della classe \cil{D}.
  }
  \bl{Coppia \cil{instanceof} + conversione}{\iz{
    \item con l'operatore \cil{instanceof} si verifica se effettivamente sia di tipo \cil{D}
    \item con la conversione si deposita il riferimento in una espressione con tipo statico \cil{D}
    \item a questo punto si può invocare il metodo
  }}
  \bl{Solo due tipi di conversione fra classi consentite}{\iz{
    \item Upcast: da sottoclasse a superclasse (spesso automatica)
    \item Downcast: da superclasse a sottoclasse (potrebbe fallire)
  }}
}

\fr{Errori possibili connessi alle conversioni}{
  \bl{Errori semantici (a tempo di compilazione, quindi innocui)}{\iz{
    \item Tentativo di conversione che non sia né upcast né downcast
    \item Chiamata ad un metodo non definito dalla classe (statica) del receiver
  }}
  \bl{Errori d'esecuzione (molto pericolosi, evitabili con l'\cil{instanceof})}{\iz{
    \item Downcast verso una classe incompatibile col tipo dinamico, riportato come \cil{ClassCastException}
  }}
  \srcode{\scriptsize}{3}{30}{\ecl/last/ShowCCE.java}
}

\fr{\Cil{instanceof}, conversioni e \Cil{Person}}{
  \sizedrangedcode{\scriptsize}{3}{100}{\ecl/person/UsePerson2.java}
}



\section{Classi astratte}

\fr{Motivazioni}{
  \bl{Fra interfacce e classi}{\iz{
    \item Le interfacce descrivono solo un contratto
    \item Le classi definiscono un comportamento completo
    \item ...c'è margine per costrutti intermedi?
  }}
  \bl{Classi astratte}{\iz{
    \item Sono usate per descrivere classi dal comportamento parziale, ossia in cui alcuni metodi sono dicharati ma non implementati
    \item Tali classi non sono istanziabili (l'operatore \cil{new} non può essere usato)
    \item Possono essere estese e ivi completate, da cui la generazione di oggetti
  }}
  \bl{Tipica applicazione: pattern Template Method}{
    Serve a dichiare uno schema di strategia con un metodo che definisce un comportamente comune (spesso \cil{final}), ma che usa metodi da concretizzare in sottoclassi
  }
}

\fr{Classi astratte}{
  \bl{Una classe astratta:}{\iz{
    \item è dichiarata tale: \cil{abstract class C ... \{\...\}}
    \item non può essere usata per generare oggetti
    \item può opzionalmente dichiarare metodi astratti: \iz{
      \item hanno forma ad esempio: \cil{abstract int m(int a, String s);}
      \item ossia senza body, come nelle dichiarazioni delle interfacce
    }
  }}
  \bl{Altri aspetti}{\iz{
    \item può definire campi, costruttori, metodi, concreti e non
    \item ...deve definire con cura il loro livello d'accesso
    \item può estendere da una classe astratta o non astratta
    \item può implementare interfacce, senza essere tenuta ad ottemperarne il contratto
    \item chi estende una classe astratta può essere non-astratto solo se implementa tutti i metodi definiti
  }}
}

\fr{Esempio: \Cil{LimitedLamp} come classe astratta}{
  \bl{Obbiettivo}{\iz{
    \item Vogliamo progettare una estensione di \cil{SimpleLamp} col concetto di esaurimento
    \item La strategia con la quale gestire tale esaurimento può essere varia
    \item Ma bisogna far sì che qualunque strategia si specifichi, sia garantito che:{\iz{
      \item la lampadina si accenda solo se non esaurita
      \item in caso di effettiva accensione sia possibile tenerne traccia ai fini della strategia
    }}
  }}
  \bl{Soluzione}{\iz{
    \item Un uso accurato di \cil{abstract}, \cil{final}, e \cil{protected}
    \item Daremo tre possibili specializzazioni per una \cil{LimitedLamp}{\iz{
      \item che non si esaurisce mai
      \item che si esaurisce all'n-esima accensione
      \item che si esaurisce dopo un certo tempo dalla prima accensione
    }}
  }}
}

\fr{UML complessivo}{
  \fg{height=0.8\textheight}{img/uml-abstract.pdf}
}

\fr{\Cil{SimpleLamp}}{
  \srcode{\scriptsize}{3}{30}{\ecl/abs/SimpleLamp.java}
}

\fr{\Cil{LimitedLamp}}{
  \srcode{\ssmall}{3}{30}{\ecl/abs/LimitedLamp.java}
}

\fr{\Cil{UnlimitedLamp}}{
  \srcode{\scriptsize}{3}{30}{\ecl/abs/UnlimitedLamp.java}
}

\fr{\Cil{CountdownLamp}}{
  \srcode{\scriptsize}{3}{30}{\ecl/abs/CountdownLamp.java}
}

\fr{\Cil{ExpirationTimeLamp}}{
  \srcode{\ssmall}{5}{30}{\ecl/abs/ExpirationTimeLamp.java}
}

\fr{\Cil{UseLamps}}{
  \srcode{\ssmall}{3}{30}{\ecl/abs/UseLamps.java}
}

\fr{Classi astratte vs interfacce}{
  \bx{\iz{
    \item Due versioni quasi equivalenti
    \item Unica differenza, ereditarietà multipla per le interfacce
  }}
  \sizedcode{\normalsize}{code/AbsInt.java}
}

\fr{Wrap-up su ereditarietà}{
  \bl{Il caso più generale:}{
    \cil{class C extends D implements I,J,K,L \{..\}}
  }
  \bl{Cosa deve/può fare la classe \cil{C}}{\iz{
    \item deve implementare tutti i metodi dichiarati in \cil{I,J,K,L} e super-interfacce
    \item può fare overriding dei metodi (non finali) definiti in \cil{D} e superclassi
  }}
  \bl{Classe astratta:}{
    \cil{abstract class CA extends D implements I,J,K,L \{..\}}
  }
  \bl{Cosa deve/può fare la classe \cil{CA}}{\iz{
    \item non è tenuta a implementare alcun metodo
    \item può implementare qualche metodo per definire un comportamento parziale
  }}
}

\section{Autoboxing dei tipi primitivi, e argomenti variabili}

\fr{Autoboxing dei tipi primitivi}{
  \bl{Già conosciamo i Wrapper dei tipi primitivi}{\iz{
    \item \cil{Integer i=new Integer(10);}
    \item \cil{Double d=new Double(10.5);}
    \item..ossia, ogni valore primitivo può essere ``boxed'' in un oggetto
  }}
  \bl{Autoboxing}{\iz{
    \item Una tecnica recente di Java
    \item Si simula l'equivalenza fra tipi primitivi e loro Wrapper
    \item Due meccanismi:{\iz{
      \item Se si trova un primitivo dove ci si attende un oggetto, se ne fa il boxing
      \item Se si trova un wrapper dove ci si attende un primitivo, si fa il de-boxing
    }}
  }}
  \bl{Risultato}{\iz{
    \item Si simula meglio l'idea ``Everything is an Object''
    \item Anche i tipi primitivi sono usabili ad esempio con \cil{Object[]}
  }}
}

\fr{Uso dell'autoboxing}{
  \srcode{\footnotesize}{3}{30}{\ecl/last/Boxing.java}
}

\fr{Variable arguments}{
  \bl{A volte è utile che i metodi abbiano un numero variabile di argomenti}{\iz{
    \item \cil{int i=sum(10,20,30,40,50,60,70);}
    \item \cil{printAll(10,20,3.5,new Object());}
    \item prima di Java 5 si simulava passando un unico array
  }}
  \bl{Variable arguments}{\iz{
    \item L'ultimo (o unico) argomento di un metodo può essere del tipo ``\cil{Type... argname}''{\iz{
      \item p.e. \cil{void m(int a,float b,Object... args)\{...\}}
    }}
    \item Nel body del metodo, \cil{argname} è trattato come un \cil{Type[]}
    \item Chi chiama il metodo, invece che passare un array, passa una lista di argomenti di tipo \cil{Type}
    \item Funziona automaticamente con polimorfismo, autoboxing, \cil{instanceof},...
  }}
}

\fr{Uso dei variable arguments}{
  \srcode{\ssmall}{3}{30}{\ecl/last/VarArgs.java}
}


\fr{Preview del prossimo laboratorio}{
  \bl{Obbiettivi}{Familiarizzare con:\iz{
    \item Estensione delle classi e corrsipondente polimorfismo
    \item Classi astratte
    \item Tipi a run-time e boxing
  }}
}


\end{document}

