\documentclass[presentation]{beamer}
\newcommand{\lectnum}{15}
\newcommand{\lectitle}{Input-Output (I/O)}
\usepackage{../oop-slides}

\begin{document}

\frame[label=coverpage]{\titlepage}

\ackpage{}


\newcommand{\ecl}{\eclipsepath{p15io}}


\fr{Outline}{
  \bl{Goal della lezione}{\iz{
  \item Illustrare le API fornite da Java per l'I/O
  \item Descrivere alcune scelte progettuali e pattern
  \item Mostrare esempi di applicazione
  }}
  \bl{Argomenti}{\iz{
  \item Classi per gestire file
  \item Classi per gestire Stream (di input/output)
  \item Serializzazione di oggetti
  \item Classi per gestire file di testo
  \item Pattern Decorator
  }}
}

\frs{5}{Il problema dell'Input/Output}{
  \bl{Uno dei problemi fondamentali per un sistema operativo}{\iz{
    \item Gestire le comunicazioni fra CPU e dispositivi affacciati sul BUS{\iz{
    \item Console, tastiera, mouse, dischi, rete, sensori, schermo
    }}
    \item Vi sono varie modalità di interazione possibili{\iz{
    \item sequenziale, random-access, buffered, per carattere/linea/byte/oggetto
    }}
    \item I sistemi operativi offrono vari meccanismi{\iz{
    \item file, I/O control interface, socket per il networking, video driver
    }}
  }}
  \bl{La libreria \cil{java.io.*}}{\iz{
    \item Fornisce i concetti di File e Stream di dati
    \item Consente una gestione flessibile dei vari aspetti
    \item È estesa con la libreria \cil{java.nio}, che vedremo poi
    \item È la base di librerie avanzate (networking,..), anche non-JDK (JSON,..)
    \item I/O con l'utente in ambiente a finestre è realizzato con le GUI
  }}
}

\fr{La libreria \Cil{java.io}}{
    \fg{height=0.85\textheight}{img/io.png}
}

\fr{I macro-elementi della libreria}{
    \bl{Outline della lezione}{\iz{
	\item File
	\item Stream di ingresso e uscita
	\item File ad accesso ``random''
	\item Stream di oggetti e serializzazione
	\item Reader e Writer di testi
    }}
}

\section[File e proprietà]{File e loro proprietà}

\fr{I File}{
    \bl{File system}{\iz{
	\item Il file system è un modulo del S.O. che gestisce la memoria secondaria
	\item Maschera le diversità di dispositivi fisici (HD, CD, DVD, BR, SSD,..)
	\item Maschera le diversità di contenuti informativi (testi, filmati, archivi,..)
	\item Fornisce meccanismi per fornire prestazioni, concorrenza, robustezza
    }}
    \bl{File}{\iz{
	\item Un file system contiene un insieme di \alert{file}
	\item Un file ha un contenuto informativo, ossia un insieme di byte{\iz{
	    \item interpretabili in vario modo (testi, programmi, strutture dati)
	    \item potrebbe essere un file virtuale, che mappa un dispositivo
	    \item un caso particolare è la directory (ossia una tabella di ID di file)
	}}
	\item Si ha una organizzazione gerarchica in cartelle (un file ha un path)
	\item Un file ha un ID, nome, percorso, diritti di accesso, dimensione, \ldots{}
    }}
}

\fr{La classe \Cil{java.io.File}}{
    \bl{Usi}{\iz{
        \item Serve a identificare un preciso file su file systems
        \item Permette di ottenere informazioni varie sul file
        \item Permette di effettuare alcune operazioni complessive (cancellazione, renaming)
        \item Permette di impostare alcune proprietà (se eseguibile, se scrivibile)
        \item Permette di ottenere informazioni generali sul file systems
        \item Permette di creare cartelle
        \item[$\Rightarrow$] non include operazioni per accedere al suo contenuto, ma vi si potrà agganciare uno stream
    }}
}       


\fr{Classe \Cil{java.io.File}: pt1}{
    \sizedrangedcode{\ssmall}{1}{28}{short/File.java}
}	

\fr{Classe \Cil{java.io.File}: pt2}{
    \sizedrangedcode{\ssmall}{30}{100}{short/File.java}
}	

\fr{\Cil{java.io.File} in azione (modella un path su File System)}{
    \sizedrangedcode{\ssmall}{7}{100}{\ecl/files/UseFile.java}
}	

\begin{frame}[fragile]\frametitle{Esempio di output}
\begin{lstlisting}[basicstyle=\ssmall\ttfamily]
getName prova.bin
getParent /home/mirko/aula/15
isAbsolute true
getCanonicalPath /home/mirko/aula/15/prova.bin
getPath /home/mirko/aula/15/prova.bin
getParentFile /home/mirko/aula/15
getAbsolutePath /home/mirko/aula/15/prova.bin
getAbsoluteFile /home/mirko/aula/15/prova.bin
getCanonicalFile /home/mirko/aula/15/prova.bin
canRead true
canWrite true
isDirectory false
isFile true
isHidden false
canExecute false
getTotalSpace 53616242688
getFreeSpace 14087458816
getUsableSpace 11357081600
getClass class java.io.File
\end{lstlisting}
\end{frame}

\fr{Accedere al contenuto di un file}{
    \bl{Come fare?}{\iz{
	\item Un file ha un contenuto informativo (potenzialmente di grosse dimensioni)
	\item Lo si potrebbe leggere (in vari modi)
	\item Lo si potrebbe scrivere (in vari modi)
	\item Il suo contenuto potrebbe essere interpretabile in vari modi
    }}
    \bl{Alcuni di tali concetti sono condivisi con altri meccanismi}{\iz{
	\item Risorse interne al classpath Java
	\item Networking e file di rete
	\item Archivi su database
	\item Depositi di informazione in memoria
    }}
    \bx{
	Il concetto di \alert{input/output-stream} è usato come astrazione unificante
    }
}

\section{Input/OutputStream}

\fr{Overview sugli \Cil{InputStream} e \Cil{OutputStream} in Java}{
    \bl{\cil{InputStream} e \cil{OutputStream}}{\iz{
	\item Stream = flusso (di dati)
	\item Di base, gestiscono flussi binari (di \cil{byte}) leggibili vs. scrivibili 
	\item Sono classi astratte (e non interfacce..)
	\item Possono essere specializzate da ``sottoclassi'' e ``decorazioni'', tra cui{\iz{
        \item Per diverse sorgenti e destinazioni di informazione, ad esempio su file (\cil{FileInputStream}) o su memoria (\cil{ByteArrayInputStream})
        \item Per diversi formati di informazione, ad esempio valori primitivi (\cil{DataInputStream}) o interi oggetti Java (\cil{ObjectInputStream})
	    \item[$\Rightarrow$]..e corrispondenti versioni \cil{Output}
	}}
    }}
    \bl{Tipicamente usati per alimentare altre classi}{\iz{
	\item File di testo (\cil{Reader}, \cil{Writer}, e specializzazioni)
% 	\item File ad accesso ``random'' (\cil{RandomAccessFile})
    \item Librerie avanzate comunemente usate per l'accesso al file system tipicamente hanno metodi che accettano (\cil{In}/\cil{Out})\cil{putStream}
    }}
}

\fr{La classe \Cil{java.io.InputStream}}{
    \sizedcode{\scriptsize}{short/InputStream.java}
}	

\fr{\Cil{FileInputStream} e \Cil{ByteArrayInputStream}}{
    \fg{height=0.85\textheight}{img/bytefile.png}
}

\fr{Uso di \Cil{ByteArrayInputStream}}{
    \bl{\cil{ByteArrayInputStream}}{\iz{
      \item crea un \cil{InputStream} a partire da un \cil{byte[]}
      \item è un wrapper
    }}
    \sizedrangedcode{\scriptsize}{3}{100}{\ecl/files/UseByteArrayStream.java}
}

\fr{Il costrutto \Cil{try-with-resources}}{
    \bl{Costrutto \cil{try-with-resources}}{\iz{
      \item vuole la creazione di un \cil{java.lang.AutoCloseable} come primo argomento
      \item ne assicura la chiusura 
      \item si possono opzionalmente aggiungere delle \cil{catch} di eccezioni
      \item andrebbe sempre usato..
    }}
    \sizedrangedcode{\scriptsize}{3}{100}{\ecl/files/UseTryWithResources.java}
}	


\fr{Esempio \Cil{StreamDumper}}{
    \sizedrangedcode{\scriptsize}{3}{100}{\ecl/files/StreamDumper.java}
}	

\fr{\Cil{UseStreamDumper} -- uso uniforme di vari \Cil{InputStream}}{
    \sizedrangedcode{\ssmall}{6}{100}{\ecl/files/UseStreamDumper.java}
}	

\fr{La classe \Cil{java.io.OutputStream}}{
    \sizedcode{\scriptsize}{short/OutputStream.java}
    \bl{Stream di uscita -- Duale all'\cil{InputStream}}{\iz{
	\item Esistono anche le analoghe specializzazioni \cil{ByteArrayOutputStream} e \cil{FileOutputStream}
    }}
}

\fr{\Cil{UseOutputStream}}{
    \sizedrangedcode{\ssmall}{3}{100}{\ecl/files/UseOutputStream.java}
}

\fr{\Cil{UseOutputStream2} -- qualche variante}{
    \sizedrangedcode{\ssmall}{3}{100}{\ecl/files/UseOutputStream2.java}
}


\fr{Salvataggio di strutture dati: \Cil{List<Byte>}}{
    \sizedrangedcode{\ssmall}{6}{100}{\ecl/files/ListOnFile.java}
}

\fr{Solo \Cil{byte}?}{
    \bl{Problema...}{\iz{
	\item Poter leggere e scrivere da uno \cil{Stream} anche \cil{int}, \cil{long}, eccetera
% 	\item Ove possibile, ciò dovrebbe essere possibile da qualunque stream
	%\item Nel caso nuove esigenze simili si debbano creare, non devono esserci interferenze o esplosione di classi
    }}
    \bl{Il concetto di decoratore}{\iz{
	\item Si definisce \cil{DataInputStream} che estende \cil{InputStream}{\iz{
	    \item e similmente \cil{DataOutputStream} che estende \cil{OutputStream}
	}}
	\item Tale nuova classe comunque fa da wrapper per un \cil{InputStream}, al quale delega le varie operazioni
	\item Un oggetto di tale nuova classe è un decoratore per quello interno.. visto che ne modifica il funzionamento
	\item Con questa tecnica è possibile decorare sia un \cil{FileInputStream} che un \cil{ByteArrayInputStream} -- o altri
    }}
}

\fr{Decorazione, in generale}{
    \fg{height=0.85\textheight}{img/decorator.png}
}

\fr{Decorazione, il caso di \Cil{DataInputStream}}{
    \fg{height=0.85\textheight}{img/dec.png}
}

\fr{\Cil{DataInputStream}}{
    \sizedrangedcode{\ssmall}{1}{100}{short/DataInputStream.java}
}

\fr{\Cil{DataOutputStream}}{
    \sizedrangedcode{\ssmall}{1}{100}{short/DataOutputStream.java}
}

\fr{\Cil{UseDataStream}}{
    \sizedrangedcode{\ssmall}{3}{100}{\ecl/files/UseDataStream.java}
}

\fr{Altra decorazione: \Cil{BufferedInputStream}, \Cil{BufferedOutputStream}}{
    \bl{Esigenza}{\iz{
	\item fornire una diversa implementazione interna dello stream
	\item non legge un byte alla volta, ma riempie un buffer
	\item questo aumenta le performance nell'accesso a file e rete
	\item come fornire la funzionalità in modo ortogonale al resto della gestione degli stream?
    }}
    \bl{\cil{BufferedInputStream}, \cil{BufferedOutputStream}}{\iz{
	\item sono ulteriori decoratori, della stessa forma dei precedenti
	\item non aggiungono altri metodi
	\item per come sono fatti i decoratori, possono essere usati in ``cascata'' a \cil{DataInputStream} e \cil{DataOutputStream}
    }}
}

\fr{\Cil{UseBufferedDataStream}}{
    \sizedrangedcode{\ssmall}{3}{100}{\ecl/files/UseBufferedDataStream.java}
}

\fr{\Cil{UseBufferedDataStream2} -- chaining dei costruttori}{
    \sizedrangedcode{\ssmall}{3}{100}{\ecl/files/UseBufferedDataStream2.java}
}


\fr{Decorazione, una visione complessiva}{
    \fg{height=0.85\textheight}{img/dec-all.png}
}

\fr{Altri decoratori di \Cil{InputStream} (..e \Cil{OutputStream})}{
    \bl{Sono molteplici, tutti usabili in combinazione}{\iz{
      \item \cil{CheckedInputStream}: mantiene un ``checksum'' per verifica integrità
      \item \cil{CipherInputStream}: legge dati poi processati dopo una cifratura
      \item \cil{DeflateInputStream}: legge dati e li comprime in formato ``deflate''
      \item \cil{InflaterInputStream}: legge dati e li scompatta dal formato ``deflate''
      \item \cil{ProgressMonitorInputStream}: legge dati con possibilità di ``unread''
    }}
}


\fr{Ancora sui decoratori}{
    \bl{Pro e contro}{\iz{
	\item Sono un mix di polimorfismo e incapsulamento
	\item Consentono di comporre funzionalità in modo piuttosto flessibile
	\item Danno luogo a più flessibilità rispetto all'ereditarietà
	\item Più complicati da usare e comprendere
    }}
    \bl{Con gli stream, è possibile comporre:}{\iz{
	\item Uno stream di sorgente dati: \cil{FileInputStream}, \cil{ByteArrayInputStream}, ..
	\item Uno (o più) stream di gestione interna: \cil{BufferInputStream}, ..
	\item Uno stream di presentazione dati: \cil{DataInputStream}, \cil{ObjectInputStream}, ..
    }}
}

\section[De/serializzazione]{De/Serializzazione di oggetti}

\fr{File ed encoding}{
    \bl{Il contenuto dei file} {
      Supponiamo utilizziate un \cil{DataOutputStream} per scrivere in sequenza i numeri da 0 a 20 (escluso)
        \iz {
            \item Che cosa avete realmente scritto?
            \item avete scritto i byte da zero a 19
            \item In esadecimale, \cil{0x000102030405060708090A0B0C0D0E0F10111213}
            \item \textbf{Non} avete scritto il testo \cil{012345678910111213141516171819}
        }
    }
    \iz {
        \item I file sono sequenze di byte
        \item Per fare input/output occorre stabilire:
        \iz {
          \item Una conversione dalla struttura dati che stiamo manipolando a sequenza di byte (encoding)
          \item Una conversione da sequenza di byte a struttura dati (decoding)
        }
        \item I \cil{Data-Stream} offrono encoding e decoding per tipi primitivi
        \item Come trattare strutture dati più articolate?
    }
}

\fr{Serializzazione di oggetti}{
    \bl{Motivazioni}{\iz{
	\item Rendere gli oggetti persistenti, e trasferibili a istanze di JVM diverse
	\item Esempio: memorizzarli su file, su array di byte, trasferirli via rete
	\item Java Serialization{\iz{
	    \item serializza con relativa semplicità strutture di oggetti anche complicate
	    \item a volte è fondamentale apportarvi correzioni ad-hoc
	}}
	
    }}
    \bl{Ingredienti Serialization}{\iz{
	    \item Interfaccia ``tag'' \cil{java.io.Serializable}
	    \item Classi \cil{ObjectInputStream} e \cil{ObjectOutputStream}
	    \item Keyword \cil{transient} per campi che non devono essere serializzati
	    \item Metodi \cil{readObject} e \cil{writeObject} (e altri) per modificare la serializzazione di default per un oggetto, o per motivi di sicurezza
	    \item Meccanismo \cil{UID} per gestire versioni diverse delle classi
    }}
}

\fr{Diagramma UML con \Cil{ObjectInputStream}}{
    \fg{height=0.85\textheight}{img/object.png}
}

\frs{5}{Classe \Cil{Person} e l'interfaccia ``tag'' \Cil{Serializable}}{
    \sizedrangedcode{\ssmall}{3}{100}{\ecl/files/Persona.java}
    \bl{\cil{Serializable} -- implementata già da molte classi, non \cil{Object}}{\iz{
	\item da implementare per avere oggetti ``automaticamente'' serializzabili
	\item ciò non comporta alcun contratto da ottemperare
    }}
}

\fr{Classe \Cil{UseObjectStream}}{
    \sizedrangedcode{\ssmall}{5}{100}{\ecl/files/UseObjectStream.java}
}

\fr{Classe \Cil{UseObjectStream}: note}{
    \bl{Note}{\iz{
	\item la \cil{writeObject()}/\cil{readObject()} fallisce se l'oggetto non è serializzabile{\iz{
	  \item se la classe dell'oggetto non implementa \cil{Serializable}
	  \item la la classe dell'oggetto ha un campo che sia un oggetto non serializzabile
	}}
	\item la \cil{readObject()} fallisce se la classe dell'oggetto non è disponibile
	\item la \cil{readObject()} fallisce se la classe dell'oggetto è una versione diversa
	
    }}
}

\fr{Come funzionano \Cil{writeObject()}/\Cil{readObject()}}{
    \bl{\cil{ObjectOutputStream.writeObject()}}{\iz{
	\item Lancia una eccezione se l'oggetto non è serializzabile
	\item Scrive sullo stream i campi dell'oggetto uno a uno \\ (di tipi primitivi o serializzabili a loro volta)
	\item Si evitano i campi con modificatore \cil{transient}
	\item In questo processo, si evita di scrivere due volte uno stesso oggetto
    }}
    \bl{\cil{ObjectInputStream.readObject()}}{\iz{
	\item Lancia una eccezione se non trova la classe o non è \alert{compatibile}
	\item Chiama il costruttore senza argomenti della prima sopra-classe non serializzabile, e da lì in giù non chiama altri costruttori
	\item Ripristina il valore dei campi leggendoli dallo stream
	\item Lascia inalterati i campi \cil{transient}
    }}
}

\fr{Il problema delle versioni di una classe: \Cil{serialVersionUID}}{
    \bl{Problema}{\iz{
	\item Si serializza un oggetto, la classe viene modificata e ricompilata, e quindi si ritira su l'oggetto..  i dati sarebbero facilmente ``corrupted''
    }}
    \bl{Soluzione: ogni classe che implementa \cil{Serializable}...}{\iz{
	\item .. deve fornire una costante ``\cil{long serialVersionUID}'' che contiene un numero univoco per quella versione della classe
	\item Se non corrisponde a quello dell'oggetto caricato si ha eccezione
    }}
    \bl{Fatti}{\iz{
	\item Se mancante Eclipse segnala warning. Può generarne uno a richiesta.
	\item Se mancante la JVM ne calcola uno suo ma è sconsigliato.
	\item Molti lasciano il campo al valore $1$, non preoccupandosene.
    }}
}

\fr{I campi \Cil{transient}}{
    \bl{I campi \cil{transient} non vengono serializzati. In quali casi servono?}{\iz{
	\item Campi aggiunti per motivi di performance (p.e., caching di un calcolo), e che quindi possono essere ricostruiti a partire dagli altri campi
	\item Campi che contengono info specifiche sul run corrente della JVM (p.e., logs), e che quindi non avrebbero più senso quando l'oggetto viene recuperato dallo stream
	\item Campi che contengono oggetti comunque non serializzabili (p.e., \cil{Object}), e che quindi porterebbero ad una eccezione
	\item Campi per i quali si vuole prevedere un meccanismo di serializzazione diverso
    }}
}

\fr{\Cil{CPersona} con caching \Cil{toString}, pt1}{
    \sizedrangedcode{\ssmall}{3}{29}{\ecl/trans/CPersona.java}
}

\fr{\Cil{CPersona} con caching \Cil{toString}, pt2}{
    \sizedrangedcode{\ssmall}{31}{100}{\ecl/trans/CPersona.java}
}

\fr{\Cil{UseTransient}}{
    \sizedrangedcode{\ssmall}{6}{100}{\ecl/trans/UseTransient.java}
}

\fr{Progettazione di una serializzazione ad-hoc}{
    \bl{Serializzazione ad-hoc}{\iz{
	\item Il modello transient/non-transient a volte non è sufficiente
	\item A volte serve serializzare in modo diverso certi campi
	\item \`E possibile definire per la classe serializzabile i metodi ``\cil{void readObject(ObjectInputStream in)}'' e ``\cil{void writeObject(ObjectOutputStream out)}''
	\item Se definiti, \cil{ObjectInputStream} e \cil{ObjectOutputStream} chiamano quelli
    }}
    \bl{Dettagli}{\iz{
	\item Tali metodi possono cominciare con la chiamata a \cil{defaultReadObject()}/\cil{defaultWriteObject()}, per leggere i campi non-statici e non-transienti
	\item Si può quindi proseguire scrivendo/leggendo negli stream in input quello che si vuole
    }}
}

\fr{\Cil{APersona}: serializzazione ad-hoc per una data}{
    \sizedrangedcode{\tiny}{6}{35}{\ecl/trans/APersona.java}
}


\fr{\Cil{UseAdHocSerialization}}{
    \sizedrangedcode{\scriptsize}{5}{100}{\ecl/trans/UseAdHocSerialization.java}
}

\fr{Serializzazione ad-hoc per \Cil{java.util.ArrayList}, pt1}{
    \sizedrangedcode{\ssmall}{1}{23}{short/ArrayList.java}
}

\fr{Serializzazione ad-hoc per \Cil{java.util.ArrayList}, pt2}{
    \sizedrangedcode{\ssmall}{25}{100}{short/ArrayList.java}
}

\fr{Note sulla serializzazione Java}{
    \bl{Raramente utilizzata in applicazioni ``vere''}{
        Applicazioni vere tendono a non utilizzare \cil{Object}(\cil{In}/\cil{Out})\cil{putStream}:
        \iz{
            \item Nessuna standardizzazione
            \item Poco efficiente in termini di spazio
            \item Dispendiosa in termini di performance
            \item Scarsa portabilità (solo da/a software Java)
            \item \cil{readObject} / \cil{writeObject} in qualche modo violano il linguaggio
        }
    }
    \bl{Importanza di \cil{Serializable}}{
        Approcci diversi alla serializzazione spesso:
        \iz{
            \item Serializzano oggetti \cil{Serializable} con campi \cil{Serializable} e non \cil{transient} (deve valere ricorsivamente)
            \item Prevedono sistemi custom per la serializzazione di oggetti diversi
            \item Non fanno uso di \cil{readObject} / \cil{writeObject}
        }
    }
}


\section{\texttt{RandomAccessFile}}

\fr{Classe \Cil{RandomAccessFile}}{
    \bl{Motivazioni}{\iz{
	\item Alcuni file sono di grosse dimensioni, e non vengo letti/scritti per intero come nei casi visti finora
	\item Bensì si modifica qualche elemento ``a metà'', o se ne aggiungono in fondo, o si legge un elemento in una data posizione
    }}
    \bl{Classe \cil{RandomAccessFile}}{\iz{
	\item Non è usata tramite \cil{InputStream} o \cil{OutputStream}
	\item Fornisce i metodi di \cil{DataInput} e \cil{DataOutput}
	\item Fornisce metodi aggiuntivi:{\iz{
	    \item \cil{getFilePointer}: torna la posizione corrente nel file
	    \item \cil{seek}: imposta la nuova posizione nel file
	    \item \cil{length}: torna la lunghezza del file
	    \item \cil{setLength}: imposta la lunghezza del file
	}}
    }}
}

\fr{Classe \Cil{RandomAccessFile}}{
    \fg{height=0.85\textheight}{img/random.png}
}

\fr{\Cil{UseRandomAccessFile}}{
    \sizedrangedcode{\ssmall}{5}{100}{\ecl/files/UseRandomAccessFile.java}
}



\section{File di testo}

\fr{Limitazione degli stream binari visti finora}{
  Finora abbiamo visto stream binari, ed in particolare di oggetti Java
  \iz {
    \item Non esiste uno standard documentato
    \iz {
      \item Gli standard descrivono in modo inequivoco un protocollo (in questo caso di de/serializzazione)
      \item Difficile scrivere librerie per leggere e scrivere in quel formato
      \item Bassa \textbf{standardizzazione}
    }
    \item Non comprensibili da applicazioni non-Java
    \iz {
      \item Oggetti serializzati non apribili da applicazioni in Javascript, Python, eccetera
      \item Non adatti e.g., per applicazioni web
      \item Bassa \textbf{portabilità}
    }
    \item Non comprensibili se aperti in formato testuale
    \iz {
      \item Non modificabili da umani
      \item Non adatti e.g., per file di configurazione
      \item Bassa \textbf{leggibilità} e quindi \textbf{intelligibilità}
    }
  }
}

\fr{Limitazione degli stream binari visti}{
  \bl{Standardizzazione e portabilità} {
    \iz {
      \item Non direttamente ascrivibili al formato binario
      \item (che comunque non aiuta...)
      \item Esistono standard portabili binari, come ProtocolBuffers
    }
  }
  \bl{Leggibilità e intelligibilità} {
    \iz {
      \item Problema comune a tutti i meccanismi finora visti
      \item Esistono standard per scrivere oggetti in formato testuale
      \item Problema da risolvere in due fasi:
      \iz {
        \item Conversione da oggetto a stringa di testo e viceversa
        \item Conversione da stringa di testo a file e viceversa
      }
    }
  }
}

\fr{Conversione da oggetto a stringa di testo e viceversa} {
  \iz {
    \item Banale nel caso in cui l'oggetto sia una stringa
    \item Molto più complicato per strutture dati arbitrarie
    \item Passo necessario verso l'interoperabilità fra linguaggi
  }
  \bl{(Alcuni) Formati standard per la conversione di oggetti in testo} {
    \iz {
      \item JavaScript Object Notation -- JSON (RFC 7159)
      \iz {
        \item Nato in seno a JavaScript (che non c'entra nulla con Java)
        \item Molto usato nel web
      }
      \item Tom's Obvious, Minimal Language -- TOML
      \iz {
        \item Particolarmente indicato per file di configurazione
      }
      \item YAML Ain't Markup Language -- YAML
      \iz {
        \item Superset di JSON dalla versione 1.2
        \item Supporto per funzioni avanzate (e.g. anchoring)
        \item Molto usato per file di configurazione complessi
      }
    }
  }
}

\frs{5}{Conversione stringa a file di testo} {
  Problema risolto con tabelle di conversione (text encoding)

  Nota: importante anche per la rappr. \textbf{in memoria} dei caratteri 
  \bl{Text Encoding} {
    \iz {
        \item ASCII (RFC 20)
        \iz {
          \item l'encoding che usa il linguaggio C per i \cil{char}
          \item 1 byte per carattere (massimo 256 caratteri)
        }
        \item UTF-8 (RFC 3629)
        \iz {
          \item Standard di fatto sul web, encoding di default in Linux
          \item encoding da usare per i sorgenti di codice
          \item da 1 a 4 byte per carattere
          \item Codifica 1.112.064 simboli
        }
        \item UTF-16 (RFC 2781)
        \iz {
          \item Encoding in memoria delle \cil{String} in Java
          \item 2 o 4 byte per carattere
          \item Codifica 1.112.064 simboli
        }
        \item ISO Latin (ISO/IEC 8859-1:1998)
        \iz {
          \item Encoding di default del testo in Windows
        }
    }
  }
}

% \fr{Encoding e interoperabilità}{
%     \bl{Everything is a sequence of bytes} {
%         \iz {
%             \item Qualunque cosa leggiate o scriviate, in fondo è una sequenza di byte
%             \item L'interpretazione di tali byte implica l'esistenza di una codifica
%             \iz {
%                 \item Per il formato testuale, un encoding
%                 \item Per formati binari personalizzati, un algoritmo di (de)serializzazione
%             }
%         }
% 
%     }
%     \bl{In questo corso} {
%         \iz {
%             \item Vedremo gli strumenti base inclusi nel JDK
%             \item Per applicazioni moderne, si prediligono formati standard:
%             \iz{
%                 \item testuali: XML, JSON, YAML...
%                 \item binari: Protocol Buffers, BSON, MessagePack, Blink protocol...
%             }
%             \item Motivo: dati accessibili per qualunque applicazione / linguaggio
%             \item In Java, sono supportati da librerie esterne al JDK (eccetto XML)
%             \item Ne vedrete un esempio d'uso in laboratorio
%         }
%     }
% }

\fr{File di testo}{
    \bl{File binari vs file di testo}{\iz{
	\item Gli stream binari sono quelli visti, sono costituiti da sequenze di byte e il modo col quale si ottengono da questi altri tipi primitivi è standard
	\item Gli ``stream di testo'' hanno problematiche diverse:{\iz{
	    \item sono sequenze di caratteri
	    \item la codifica (UTF-8, UTF-16, ASCII) potrebbe variare 
	    \item la codifica potrebbe dettare anche i codici speciali di ``a capo'' etc.
	}}
	\item Questa gestione richiede classi specifiche
    }}
    \bl{\Cil{Reader} (e \Cil{Writer)}}{\iz{
	\item \cil{Reader}: la radice, con metodi per leggere caratteri, e linee di testo
	\item \cil{StringReader}: decoratore per prelevare da \cil{String}
	\item \cil{BufferedReader}: decoratore per ottimizzare gli accessi
	\item \cil{InputStreamReader}: decoratore che incapsula un \cil{InputStream}
	\item \cil{FileReader}: sua estensione per leggere da file via \cil{FileInputStream}
    }}
}

\fr{UML classi relative ai reader (writer analoghi)}{
    \fg{height=0.85\textheight}{img/text.png}
}

\fr{Esempio: \Cil{UseReadersWriters}}{
    \sizedrangedcode{\ssmall}{5}{100}{\ecl/files/UseReadersWriters.java}
}

\fr{Esempio: \Cil{UseStreamReadersWriters}}{
    \sizedrangedcode{\ssmall}{5}{100}{\ecl/files/UseStreamReadersWriters.java}
}

\fr{Il caso di: \Cil{System.in} e \Cil{System.out}}{
    \sizedrangedcode{\ssmall}{5}{100}{\ecl/files/SystemInOut.java}
}

\frs{5}{Riassunto classi}{
  \bl{Identificazione di un file (o directory)}{\iz{
    \item \cil{File}
  }}
  \bl{Accesso random}{\iz{
    \item \cil{RandomAccessFile}
  }}
  \bl{Lettura di file dati (scrittura duale)}{\iz{
    \item \cil{FileInputStream} + \cil{BufferedInputStream} + \cil{DataInputStream}
    \item \cil{FileInputStream} + \cil{BufferedInputStream} + \cil{ObjectInputStream}
  }}
  \bl{Lettura di file di testo (scrittura duale)}{\iz{
    \item \cil{FileReader} + \cil{BufferedReader}
    \item \cil{FileInputStream} + \cil{InputStreamReader} + \cil{BufferedReader}
  }}  
}



\end{document}
