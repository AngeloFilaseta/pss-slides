\documentclass[presentation]{beamer}
\newcommand{\lectnum}{00}
\newcommand{\lectitle}{Introduzione al corso}
\usepackage{../oop-slides}

\begin{document}

\frame[label=coverpage]{\titlepage}

\ackpage{}

\section{Introduzione al corso}

\frs{5}{Docenti}{
  \bl{Titolare del corso: Prof. Roberto Casadei}{
    \iz{ 
      \item e-mail | \texttt{roby.casadei@unibo.it}
      \item homepage | \url{https://www.unibo.it/sitoweb/roby.casadei/}
    }
  }
  \bl{Modulo 2: Prof. Danilo Pianini}{
    \iz{ 
      \item e-mail | \texttt{danilo.pianini@unibo.it}
      \item homepage | \url{https://www.unibo.it/sitoweb/danilo.pianini/}
    }
  }
  \bl{Modulo 3: Prof. Sara Montagna}{
    \iz{ 
      \item e-mail | \texttt{sara.montagna@unibo.it}
      \item homepage | \url{https://www.unibo.it/sitoweb/sara.montagna/}
    }
  }
  
  \bl{Tutor: Dott. Francesco Dente}{
    \iz{ 
      \item e-mail | \texttt{francesco.dente@studio.unibo.it}
    }
  }
}

\fr{Contatti con gli studenti}{
  \bl{Chi contattare}{\iz{
  \item Mirko Viroli: ``teoria'' + organizzazione del corso
  \item Pianini/Croatti/Casadei: ``pratica'' + laboratorio
  }}
  \bl{Attraverso il forum}{\iz{ 
  \item Per domande la cui risposta è di interesse generale \\ (e quindi tutte le domande tecniche)
  }}
  \bl{Via mail al docente}{\iz{ 
  \item Per questioni personali
  \item Per domande che consentono risposte concise
  }}
  \bl{Ricevimento}{\iz{ 
  \item Mirko Viroli: Mercoledì 16-18 \\ (riverificare nelle Home Page del docente)
}}}

\fr{Sito del corso}{
  \bl{Sito e-learning di Ateneo}{\iz{
  \item \myurl{https://iol.unibo.it/course/view.php?id=40399}{\iz{
    \item Short: \myurl{http://bit.ly/oop19-cesena}
    \item in evidenza della home page del docente
  }}
  \item sarà il luogo degli avvisi (e notifiche), forum di discussione, produzione di materiale
  \item tutti gli studenti che seguono il corso si iscrivano, e lo tengano d'occhio
  \item in particolare: portando a lezione le slide stampate
  }}
  \bl{Sito sul Portale APICE: un semplice ``entry point''}{\iz{
  \item \myurl{http://apice.unibo.it/xwiki/bin/view/Courses/OOP1920}{\iz{
    \item Link in evidenza a partire da: \myurl{http://mirkoviroli.apice.unibo.it}
  }}
  }}
}


\fr{Organizzazione generale del corso}{
  \bl{Lezioni aula (due da 3 ore la settimana)}{\iz{ 
  \item Illustrano i concetti teorici, metodologici e pratici
  \item Basate su slide proiettate (ma non solo)
  }}
  \bl{Laboratorio (turni da 3-4 ore a settimana)}{\iz{
  \item A giorni alterni (il che vi lascia un giorno libero)
  \item Illustra ulteriori aspetti metodologici e pratici
  \item Con esercizi necessari alla comprensione e alla sperimentazione
  \item \alert{\`E parte integrante del corso}
  }}
  \bl{Studio a casa (almeno 4 ore a settimana) -- p.e., nel giorno libero}{\iz{
  \item Rilettura slide, esperimenti pre- e/o post-laboratorio
  \item È praticamente obbligatorio se volete rimanere in pari..
}}}

\fr{Programma (di massima) del corso}{
  \bl{Parti principali}{\iz{ 
  \item Elementi base di programmazione OO e Java
  \item Polimorfismo (ereditarietà, subtyping, genericità)
  \item Librerie (I/O, grafica, concorrenza)
  \item Integrazione col paradigma funzionale (Java 8 lambda)
  \item Pattern e buone pratiche di programmazione
  \item Elementi di programmazione C\#
}}}

\fr{Testi di riferimento (non necessario l'acquisto)}{
  \bl{Programmazione in Java}{\iz{ 
  \item B.Eckel. Thinking in Java, 4th edition.
  \item J.Block. Effective Java, 2nd edition.
  \item R.Warburton. Java 8 Lambdas.
  }}
  \bl{Programmazione in C\#}{\iz{ 
  \item Jon Skeet. C\# in depth, 3rd edition.
  }}
  \bl{Altri riferimenti}{\iz{
  \item E.Gamma et al. Design Patterns Elements of Reusable Object-Oriented Software.
  \item R.Martin. Clean Code: A Handbook of Agile Software Craftsmanship
  \item Java e C\# online documentation (tutorials, Language Specification, APIs)
} }}

\frs{15}{Software}{
  \bl{Java}{\iz{ 
  \item OpenJDK 11 (Open Java Development KIT){\iz{
    \item \myurl{https://openjdk.java.net} o \myurl{https://adoptopenjdk.net/}
    }}
  \item Eclipse IDE (2019 06), e correlati
strumenti:{\iz{
      \item Git,
      \item PMD Eclipse plugins (PMD, Checkstyle, FindBugs)
  }}}}
  \bl{Altri framework -- parte finale del corso}{\iz{ 
  \item Visual Studio .NET (Microsoft)
  \item framework Mono e IDE Monodevelop/Raider (Linux e MacOs)
  }}  
  \bl{Istruzioni sull'installazione (sul PC di casa)}{\iz{ 
  \item Già disponibili sul sito e-learning del corso
  \item \myurl{https://oop-at-disi.github.io/software-installation-instructions/}
  \item Molto importante rendersi operativi a casa nel giro di una settimana!
  \item L'uso di Eclipse sarà necessario solo fra 2/3 settimane
  \item[$\Rightarrow$] sarebbe consigliato l'uso del sistema operativo Linux..
  }}
}

\fr{Sul ruolo di questo corso}{
    \bl{Elementi essenziali}{\iz{
    \item Costruzione del software, e quindi di sistemi
    \item Analisi problemi, e organizzazione di soluzioni
    \item Tecniche base ed (alcune) avanzate di programmazione ad oggetti
    \item Introduzione al trend della programmazione moderna
    \item Gestione del progetto
    \item Utilizzo di strumenti integrati di sviluppo
    }}
    \bl{L'importanza nel vostro percorso}{\iz{
    \item Enfasi sull'approccio metodologico
    \item Competenze a curriculum
    \item Target di qualità piuttosto elevato
    \item È cruciale dedicargli subito il tempo necessario
    }}
}

\frs{5}{Esame}{
  \bl{Prova scritta}{\iz{ 
  \item Durata 1.5 ore circa, svolta in laboratorio
  \item Verifica mirata di capacità tecniche, di problem-solving, buona progettazione OO
  \item[$\Rightarrow$] Forniremo esercizi-tipo in laboratorio
  \item[$\Rightarrow$] I temi d'esame dello scorso a.a. sono disponibili e abbastanza indicativi
  }}
  \bl{Discussione progetto}{\iz{
  \item Progetto sviluppato in gruppo, 70-80 ore a testa (vedi i 12 CFU)
  \item Concordato col docente prima di iniziare
  \item Da relazionare con qualità, poi discusso su appuntamento
  \item Consegne con deadline scelte da voi ma a quel punto stringenti: \\ orientativamente, fine Febbraio, Aprile, Giugno, Agosto, Settembre
  \item In discussione si porta anche un micro-task C\# ($<$ 10 ore)
  \item[$\Rightarrow$] I dettagli discussi a metà corso
    \item[$\Rightarrow$] Regole d'esame già disponibili sul sito e-learning

}}
}

\fr{Prerequisiti}{
    \bl{Buona conoscenza}{\iz{
    \item tecniche di programmazione imperativa/strutturata
    \item costruzione e comprensione di semplici algoritmi e strutture dati
    }}
    \bl{Attenzione a chi è già ``fluente'' in linguaggi ad oggetti}{\iz{
    \item Java o C\#
    \item è difficile disimparare le ``bad practice''..
    }}
}



\end{document}