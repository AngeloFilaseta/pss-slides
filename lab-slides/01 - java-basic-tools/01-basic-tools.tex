%\documentclass[12pt,handout]{beamer}
\documentclass[xcolor=dvipsnames,presentation]{beamer}
\usepackage{../oop-slides-lab}
\setbeamertemplate{bibliography item}[text]

\newcommand{\lab}{Lab01}

\title%[{\lab} -- Strumenti di Base]
{Strumenti di Base}

\date[\today]{\today}

\begin{document}

\frame[label=coverpage]{\titlepage}

\begin{frame}<beamer>
    \frametitle{Outline}
    \tableofcontents[]
\end{frame}

\section{Java Development Kit (JDK) Overview}

\fr{Java Development Kit (JDK)}{
    \bl{Esistono diverse distribuzioni di Java}{
        \begin{description}
            \item[JRE] -- Java Runtime Environment: solo interprete
            \item[JDK] -- JRE + tool per lo sviluppo di programmi (e.g. compilatore)
            \item[Java EE (dismesso)] -- Java Enterprise Edition: JDK + set esteso di librerie
        \end{description}
    }

    \bl{Noi faremo riferimento al JDK}{Include il necessario per eseguire applicazioni
    Java (JRE), i tool (e molte librerie) per sviluppare applicazioni,
    e la relativa  documentazione}

    \bl{I principali (non tutti) tool del JDK}{
        \begin{description}
            \item[javac] -- Compilatore Java
            \item[java] -- Java virtual machine, per eseguire applicazioni Java
            \item[javadoc] -- Per creare automaticamente documentazione in HTML
            \item[jar] -- Creazione di archivi compressi (anche eseguibili)
        \end{description}
    }
}

\fr{Java Development Kit (JDK)}{
    \begin{itemize}
        \item Java 11 -- (pen)ultima versione LTS del JDK
        \begin{itemize}
            \item Esistono varie versioni e vari distributori (OpenJDK, Oracle JDK, Eclipse OpenJ9, Amazon, Graal, Zulu\dots)
            \item OpenJDK -- implementazione non commerciale, reference version
            \begin{itemize}
                \item Per capire se una JVM si comporta correttamente, si confronta il suo comportamento con OpenJDK.
            \end{itemize}
        \end{itemize}
    \end{itemize}
    \begin{block}{Riferimenti per il corso}
        \begin{itemize}
            \item Java 11, distribuita da OpenJDK
            \item Adoptium: riferimento per il download del JDK
            \begin{itemize}
                \item \texttt{https://adoptium.net/}
            \end{itemize}
            \item Chi lo desidera può sperimentare con le versioni più recenti
            \begin{itemize}
            \item L'ultima al momento è Java 17, nuova LTS
            \item Attenzione: la compatibilità è solo all'indietro
            \item \dots{}e spesso non completa\dots{}
            \end{itemize}
        \end{itemize}
    \end{block}
}

%====================
%Architettura a Runtime
%====================
\fr{Java Virtual Machine: Architettura a Runtime}{
    \fg{height=0.80\textheight}{img/arch_runtime.png}
}

\section{Il file system e l'interprete dei comandi (richiamo)}

\begin{frame}{Elementi base del file system}
  \begin{itemize}
    \item I sistemi operativi odierni consentono di memorizzare permanentemente le informazioni su
supporti di memorizzazione di massa (dischi magnetici, dispositivi a stato solido...)
    \item Le informazioni su questi supporti sono organizzate in file e cartelle:
      \begin{itemize}
        \item i file contengono le informazioni
        \item le cartelle sono contenitori, all'interno contengono i file ed altre cartelle
      \end{itemize}
    \item La cartella più esterna, che contiene tutte le altre, è detta root. Essa rappresenta il livello gerarchico più alto del file system
      \begin{itemize}
        \item In *nix (Linux, MacOS, BSD, Solaris...), vi è una unica radice, ossia \texttt{/}
        \item In Windows, c'è una root per file system identificata da una lettera di unità (e.g.
\texttt{C:}, \texttt{D:})
      \end{itemize}
    \item La stringa che descrive un intero percorso dalla root fino ad un elemento del file system
prende il nome di \emph{percorso} (o \emph{path}), e.g.:
    \begin{itemize}
        \item \texttt{/home/user/frameworkFS.jar} (percorso Unix)
        \item \texttt{C:\textbackslash{}Windows\textbackslash{}win32.dll} (percorso Windows)
    \end{itemize}
  \end{itemize}
\end{frame}

\begin{frame}[fragile]{Manipolare il file system}
  L'utente può osservare e manipolare il file system:
  \begin{itemize}
    \item sapere quali file e cartelle contiene una cartella
    \item creare nuovi file e cartelle
    \item spostare file e cartelle dentro altre cartelle
    \item rinominare file e cartelle
    \item eliminare file e cartelle
  \end{itemize}
  Il software che consente di osservare e manipolare il file system prende il nome di \alert{file manager}.
  \begin{itemize}
    \item Su Windows, esso è ``Esplora risorse'' (\texttt{explorer.exe})
    \item Su MacOS, il principale è ``Finder''
    \item Su Linux ne esistono diversi (Nautilus, Dolphin, Thunar, Astro\ldots)
  \end{itemize}
\end{frame}

\fr{Interprete Comandi}{
  \bl{}{
    Programma che permette di interagire con il S.O. mediante comandi impartiti in modalità testuale (non grafica), via linea di comando
  }
  \begin{itemize}
    \item Nell'antichità (in termini informatici) le interfacce grafiche erano sostanzialmente inesistenti,
    e l'interazione con i calcolatori avveniva di norma tramite interfaccia testuale
    \item Tutt'oggi, le interfacce testuali sono utilizzate:
    \begin{itemize}
      \item per \textbf{automatizzare} le operazioni
      \item per \textbf{velocizzare} le operazioni (scrivere un comando è spesso molto più veloce di andare a fare click col mouse in giro per lo schermo)
      \item per fare operazioni complesse con pochi semplici comandi
      \item non tutti i software sono dotati di interfaccia grafica
      \item alcune opzioni di configurazione del sistema operativo restano accessibili solo via linea di comando
      \begin{itemize}
        \item (anche su Windows: ad esempio i comandi per associare le estensioni ad un eseguibile)
      \end{itemize}
    \end{itemize}
  \end{itemize}
  \bl{}{
    Lo vedrete in maniera esaustiva nel corso di Sistemi Operativi...
  }
}

\fr{Sistemi *nix (Linux, MacOS X, FreeBSD, Minix...)}{
    \bl{Nei sistemi UNIX-like esistono vari tipi di interpreti, chiamati shell}{
        Alcuni esempi
        \iz{
            \item Bourne shell (sh)
            \iz{ \item Prima shell sviluppata per UNIX (1977)}
            \item C-Shell (csh)
            \iz{ \item Sviluppata da Bill Joy per BSD}
            \item Bourne Again Shell (bash)
            \iz{ \item Parte del progetto GNU, è un super set di Bourne shell}
            \item ZSH, Fish, e altri terminali di ultima generazione
            \iz{
                \item Altamente personalizzabili
                \item Molto flessibili
                \item Autocompletamento avanzato e contestualità
            }
        }
    }
    \bl{Per una panoramica completa delle differenze}{
        \textcolor{blue}{\url{http://www.faqs.org/faqs/unix-faq/shell/shell-differences/}}
    }
}

\fr{Sistemi Windows}{
    \bl{}{
        L'interprete comandi è rappresentato dal programma \texttt{cmd.exe} in \texttt{C:$\backslash$Windows$\backslash$System32$\backslash$cmd.exe}
        \iz{
            \item Eredita in realtà sintassi e funzionalità della maggior parte dei comandi del vecchio MSDOS
            \item Versioni recenti hanno introdotto PowerShell, basato su .NET e C\#
            \item Da Windows 10 è possibile installare Linux dentro Windows usando Windows Subsystem for Linux (WSL2)
            \begin{itemize}
                \item Può essere un modo ragionevole di avere shell Unix in ambiente Windows
                \item %\dots{}e infatti 
                in laboratorio è disponibile una WSL2 con zsh $\heartsuit$
            \end{itemize}
        }
    }
}

\begin{frame}[fragile, allowframebreaks]{Aprire un terminale in laboratorio}
    \begin{block}{Terminale DOS/cmd}
        \begin{itemize}
            \item Start $\Rightarrow$ Programmi $\Rightarrow$ Accessori $\Rightarrow$ Prompt dei comandi
            \item Oppure: Start  $\Rightarrow$ Nella barra di ricerca, digitare \texttt{cmd} $\Rightarrow$ cliccare su \texttt{cmd.exe}
        \end{itemize}
    \end{block}
    \begin{block}{Terminale Bash emulato (git bash)}
        \begin{itemize}
            \item Icona sul desktop
            \item Oppure: Start  $\Rightarrow$ Nella barra di ricerca, digitare \texttt{git bash}
            \item Consente di lavorare su Windows con un terminale meglio equipaggiato, vicino ad un nativo Linux
        \end{itemize}
    \end{block}
    \begin{block}{Manjaro con Terminale ZSH via WSL2 (con configurazioni di Pianini)}
        \begin{itemize}
            \item Avviare WSL tramite collegamento ``Lancia WSL'' sul desktop
            \item Attendere l'importazione della distro Linux (qualche minuto)
            \item Non aprire altre istanze finché la prima non è avviata!
            \item Questa è una shell nativa Linux, preconfigurata dal docente
            \begin{itemize}
                \item (e quindi potenzialmente con personalizzazioni che dipendono dal gusto personale)
            \end{itemize}
        \end{itemize}
    \end{block}
\end{frame}

\begin{frame}[fragile]{File system e terminale: cheat sheet}
\label{slide:commands}
  \begin{center}
    \begin{tabular}{| l | c | c |}
      \hline
      \textbf{Operazione} & \textbf{Comando Unix} & \textbf{Comando Win} \\ \hline
      \scriptsize{}Visualizzare la directory corrente & \texttt{pwd} & \texttt{echo \%cd\%}  \\ \hline
      \scriptsize{}Eliminare il file \texttt{foo} (non va con le cartelle!) & \texttt{rm foo} & \texttt{del foo} \\ \hline
      \scriptsize{}Eliminare la directory di nome \texttt{bar} & \texttt{rm -r bar} & \texttt{rd bar} \\ \hline
      \scriptsize{}Contenuto della directory corrente & \texttt{ls -alh} & \texttt{dir} \\ \hline
%       \scriptsize{}Avviare un eseguibile di nome \texttt{pippo} & \texttt{./p} & \texttt{.\textbackslash{}p} \\ \hline
      \scriptsize{}Cambiare unità disco (passare a D:) & -- & \texttt{D:} \\ \hline
      \scriptsize{}Passare alla directory di nome \texttt{baz} & \texttt{cd baz} & \texttt{cd baz} \\ \hline
      \scriptsize{}Passare alla directory di livello superiore & \texttt{cd ..} & \texttt{cd..} \\ \hline
      \scriptsize{}Spostare (rinominare) un file \texttt{foo} in \texttt{bar} & \texttt{mv foo bar} & \texttt{move foo bar} \\ \hline
      \scriptsize{}Copiare il file \texttt{foo} in \texttt{baz} & \texttt{cp foo bar} & \texttt{copy foo bar} \\ \hline
      \scriptsize{}Creare la directory di nome \texttt{bar} & \texttt{mkdir bar} & \texttt{md bar} \\ \hline
    \end{tabular}
  \end{center}
%  Eseguire delle prove ed esser certi di aver compreso come utilizzare ogni comando. Per \emph{cominciare} l'esame, in particolare, dovrete usare il comando \texttt{cd}: siate certi di aver capito cosa fa!
\end{frame}

\begin{frame}[fragile, allowframebreaks]{Uso intelligente del terminale}
    I terminali moderni possono essere utilizzati in modo piuttosto efficace.
    Il numero ed il tipo di utilizzi "intelligenti" cambia da terminale a terminale e da sistema a sistema.
    \begin{block}{Autocompletamento}
        \scriptsize{}
        Sia *nix che Windows offrono la possibilità di effettuare autocompletamento, ossia chiedere al sistema di provare a completare un comando.
        \begin{itemize}
            \item Per farlo si utilizza il tasto \enf{``TAB''}.
            \item L'autocompletamento cambia anche molto fra terminali diversi
            \begin{itemize}
                \scriptsize{}
                \item Zsh e Fish in particolare posso ``capire'' come autocompletare i sottocomandi in modo intelligente.
            \end{itemize}
        \end{itemize}
    \end{block}
    \begin{block}{Memoria dei comandi precendenti}
        \scriptsize{}
        Sia Bash/zsh che cmd offrono la possibilità di richiamare rapidamente i comandi inviati precedentemente premendo il tasto \enf{``FRECCIA SU'' ($\uparrow$)}.
        \begin{itemize}
            \item  Su bash/zsh i comandi sono \emph{persistenti} (disponibili anche se il terminale viene riavviato).
        \end{itemize}
    \end{block}
    \begin{block}{Interruzione di un programma}
        \scriptsize{}
        È possibile interrompere forzatamente un programma, qualora non risponda per vari motivi o non si voglia attenderne la normale terminazione.
        \begin{itemize}
            \item Per farlo, sia su Windows che in *nix, si prema \enf{CTRL+C}.
            \item Esempio tipico: programma in loop infinito
        \end{itemize}
    \end{block}
    \begin{block}{Visualizzazione della storia dei comandi}
        \scriptsize{}
        \begin{itemize}
             \item Su bash / zsh / fish eccetera, usare il comando \texttt{history}
            \item Su Windows cmd, usare il tasto F7
        \end{itemize}
    \end{block}
    \begin{block}{Ricerca nella storia dei comandi precedenti}
        \scriptsize{}
        \begin{itemize}
            \item Premendo CTRL+R seguito da un testo da cercare, i sistemi *nix supportano la ricerca all'interno dei comandi lanciati recentemente, anche in sessioni utente precedenti. Non disponibile su Windows.
        \end{itemize}
    \end{block}
\end{frame}



\section{Compilazione ed Esecuzione da Riga di Comando}

\fr{Compilazione ed Esecuzione, comandi di base}{
    \bl{Compilazione}{
        Compilazione di una classe (comando \textcolor{blue}{javac})
        \iz{
            \item \alert{\cil{javac NomeFileSorgente.java}}
            \item \alert{\cil{javac *.java}} compila tutti i sorgenti nella directory corrente
        }
    }
    \bl{Esecuzione}{
        Esecuzione di un programma Java (comando \textcolor{blue}{java})
        \iz{
            \item    \alert{\cil{java NomeDellaClasse}}
        }
    }
    Si noti che il compilatore \emph{traduce} \textbf{file} sorgente in \textbf{file} binari,
    mentre l'interprete Java (la Java Virtual Machine) esegue una ed una sola \textbf{classe}.

    Il compilatore Java lavora su \textit{file}, l'interprete Java su \textit{classi}.
}

\begin{frame}{Da Java 9: interpretazione ``al volo''}
    A partire da Java 9, è stata introdotto in java l'interprete \textit{REPL} \texttt{jshell}
    \begin{itemize}
        \item Sta per \textbf{R}ead-\textbf{E}val-\textbf{P}rint \textbf{L}oop
    \end{itemize}
    \texttt{jshell} consente di effettuare al volo compilazione ed esecuzione
    \begin{itemize}
        \item Lo useremo ogni tanto per mostrarvi il risultato di alcune espressioni
        \item Non utile per lo sviluppo di intere applicazioni
    \end{itemize}
    In sistemi con \texttt{jshell}, lanciare l'interprete \texttt{java} su un file non compilato
    \begin{itemize}
        \item Ad esempio, \texttt{java HelloWorld.java}
    \end{itemize}
    può causare l'esecuzione!
    \begin{block}{Dietro le quinte}
        \textit{jshell} non è magico: dietro le quinte compila in memoria, quindi lancia l'interprete sul bytecode risultante
    \end{block}
    Noi costruiamo applicazioni: \textbf{prima} \textit{si compila}, \textbf{poi} \textit{si esegue}!
\end{frame}

\section{Preparazione dell'ambiente di lavoro (per Lab01)}

\fr{Preparazione Ambiente di Lavoro}{
    \iz{
        \item Accendere il PC ed eseguire l'accesso
        \item Accedere al sito del corso
        \item Scaricare dalla sezione dedicata a questa settimana il materiale dell'esercitazione odierna
        \item Spostare il file scaricato sul Desktop
        \item Decomprimere il file sul Desktop
    }
}


\fr{Preparazione Ambiente di Lavoro}{
    \iz{
        \item Aprire un terminale
        \item Si scelga l'opzione che si preferisce fra quelli proposti
        \item Si comprenda in quale cartella ci si trova usando l'apposito comando
        \iz{
            \item Generalmente il prompt si apre nella directory principale dell'utente
            \item Nel caso in cui si usasse zsh dentro WSL, o Git Bash, il file system di windows è agganciato emulando un file system Unix.
        }
        \item Posizionarsi all'interno della cartella scompattata con l'ausilio del comando \ttb{cd} (change directory)
        \item Verificare che il contenuto della cartella sia quello atteso
    }
}

\begin{frame}{Preparazione Ambiente di Lavoro}
    \begin{itemize}
        \item Scegliere un editor di testo per modificare i file sorgente
        \begin{itemize}
            \item Il laboratorio è equipaggiato con diversi editor di testo: JEdit, Notepad++, Visual Studio Code...
            \item \textbf{NON} sono adatti per l'apertura e la modifica di file Java
            (né di alcun altro linguaggio di programmazione) i word processors (Libreoffice Writer, Microsoft Word, Microsoft WordPad...), né l'editor di testo incluso in Windows (Notepad).
        \end{itemize}
        \item Si suggerisce l'utilizzo di \textbf{Code} (\texttt{https://code.visualstudio.com/})
        \begin{itemize}
            \item fornisce un supporto adeguato alla scrittura/modifica di sorgenti Java
            \item offre la possibilità di interpretare la sintassi markdown: modalità con cui sono forniti i testi degli esercizi (per i primi laboratori)
        \end{itemize}
    \end{itemize}
\end{frame}

\section{Appendice: richiami utili per gli esercizi del Lab01}

\fr{A1 -- Operazioni con i Numeri Complessi}{
    \bl{Numeri complessi -- breve ripasso}{
        Siano $z,w \in \mathbb{C} : z = a + ib,\ w = c + id$, allora:
        \iz {
            \item Confronto: $z = w \iff a = c \wedge b = d$
            \item Somma algebrica: $z \pm w = a \pm c + i(b \pm d)$
            \item Prodotto: $z \cdot w = (a+ib)(c+id) = ac-bd+i(bc+ad)$
            \item Rapporto: $\frac{z}{w} = \frac{(a+ib)(c-id)}{(c+id)(c-id)} = \frac{ac + bd}{c^2 + d^2} + i\frac{bc - ad}{c^2 + d^2}$
        }
    }
}

\end{document}

\begin{frame}[allowframebreaks]
 \frametitle{Bibliography}
    \bibliographystyle{plain}
    \small
 \bibliography{biblio}
\end{frame}

